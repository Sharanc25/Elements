\chapter{Literature Review}
Titanium can exist in two allotropic forms: alpha (a hexagonal close-packed crystal structure) and beta (a body-cen- tered cubic structure) (Ref 7.1–7.4). In pure tita- nium, the alpha (a) phase is stable up to 880 °C (1620 °F), at which point it transforms to the beta (b) phase; the beta phase is stable from 880 °C (1620 °F) to the melting point. At room temperature, pure titanium consists of the alpha phase. However, the alloys can contain alpha, mixtures of alpha and beta, or beta phases, de- pending on the alloy content and conditions. Thus, the alloys are classified into these structural types: alpha ($\alpha$), alpha-beta ($\alpha$-$\beta$), and beta ($\beta$).


There are two major breakdowns in classifying
the alloying elements. These are based on: A- whether or not the alloying elements are between the titanium atoms (interstitial) or replace titanium atoms (substitutional), and B-whether the alpha phase is entered preferentially (alpha stabilizing) or the beta phase is entered preferentially (beta stabilizing). The beta stabilizing elements generally are classified further, depending on whether or not there is a continuous series of solid solution between the alloying element and beta titanium (beta iso- morphous), or the beta phase decomposes eutec- toidally (beta eutectoid).