\documentclass[11pt,letterpaper]{article}

\newcommand{\workingDate}{\textsc{2017 $|$ August $|$ 11}}
\newcommand{\userName}{Sharan Chandran}
\newcommand{\institution}{IISc, Bangalore}
\usepackage{researchdiary_png}
\usepackage{multirow}
\usepackage{graphicx}
\usepackage{grffile} %% For multidot filenames - https://tex.stackexchange.com/questions/110513/unknown-graphics-extension-1-png
\usepackage{caption}
\usepackage{subcaption}
\usepackage{epstopdf}
\usepackage{float} %% For image placement
\usepackage{amsmath}
\usepackage{hyperref} % To make links clickable
\usepackage{xcolor}   % To remove ugly borders on links - https://tex.stackexchange.com/questions/823/remove-ugly-borders-around-clickable-cross-references-and-hyperlinks
\hypersetup{
    colorlinks,
    linkcolor={red!50!black},
    citecolor={blue!50!black},
    urlcolor={blue!80!black}
}
\usepackage{inputx} % Make relative paths work for \include command - https://tex.stackexchange.com/questions/4602/how-to-make-the-main-file-recognize-relative-paths-used-in-the-imported-files





\begin{document} 
%% Adds images directory for each chapter
\graphicspath{{../Images/},{../Images/Ti-6246/Microstructures/Edited/},{../Images/Ti-6246/Photos/},{../Images/Ti-6242/Photos/},{../Images/Ti-6242/Microstructures/Edited/}} 

%Relative path for \include commands. From inputx custom package.
\inputpaths{Month} 

\iffalse %Works only with input command
%http://tex.stackexchange.com/questions/8351/what-do-makeatletter-and-makeatother-do
\makeatletter % changes the catcode of @ to 11
\def\input@path{{../Month/}}
\makeatother % changes the catcode of @ back to 12
\fi

%\univlogo
{\Huge Primary Alpha, Transformed Beta and Low Cycle Fatigue Studies for Ti-6242 and Ti-6246 Alloys}\\[5mm]



\renewcommand\contentsname{whatever}
\tableofcontents

\title{Research Log}

% Using Include to improve compilation speed. Cannot Nest Include files. - https://tex.stackexchange.com/questions/246/when-should-i-use-input-vs-include

\include{"2017-08-[August]"}

\include{"2017-09-[September]"}

\end{document}