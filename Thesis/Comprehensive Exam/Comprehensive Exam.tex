\documentclass[10pt]{beamer}

\usetheme[progressbar=frametitle]{metropolis}
\usepackage{appendixnumberbeamer}

\usepackage{booktabs}
\usepackage[scale=2]{ccicons}

\usepackage{pgfplots}
\usepgfplotslibrary{dateplot}

\usepackage{xspace}
\newcommand{\themename}{\textbf{\textsc{metropolis}}\xspace}

\usepackage{amsmath}
\usepackage{graphicx}

\usepackage{pgfgantt} % To make Gantt Charts

\title{Primary Alpha, Transformed Beta and Low Cycle Fatigue of Titanium 6246 and 6242}
\subtitle{Comprehensive Exam}
\date{\today}
\date{}
\author{Sharan Chandran}
\institute{Indian Institute of Science}
% \titlegraphic{\hfill\includegraphics[height=1.5cm]{logo.pdf}}

\begin{document}

\metroset{block=fill}

%%%%%%%%%%%%%% For Gantt Chart %%%%%%%%%%%%%%%%%%%
\definecolor{barblue}{RGB}{153,204,254}
\definecolor{groupblue}{RGB}{51,102,254}
\definecolor{linkred}{RGB}{165,0,33}
\renewcommand\sfdefault{phv}
\renewcommand\mddefault{mc}
\renewcommand\bfdefault{bc}
\setganttlinklabel{s-s}{START-TO-START}
\setganttlinklabel{f-s}{FINISH-TO-START}
\setganttlinklabel{f-f}{FINISH-TO-FINISH}
\sffamily

\maketitle

\begin{frame}{Table of contents}
  \setbeamertemplate{section in toc}[sections numbered]
  \tableofcontents[hideallsubsections]
\end{frame}

\section{Objectives}

\begin{frame}[fragile]{Statement of work}

\begin{enumerate}
\item 3 heat treatments in alpha + beta and beta processed condition to generate different statistics of microstructure and microtexture.
\item Large and local area EBSD to generate
\begin{enumerate}
\footnotesize
\item adequate statistical description of primary alpha grains in terms of alpha/alpha grain contact
\item the statistics of primary grains that are BOR related to the surrounding beta
\item secondary alpha texture and its correlation with primary alpha texture
\item statistics of colony size and basket weave group size  for different heat treatments
\end{enumerate}
\item Generation of LCF data: 
\begin{enumerate}
\color{red}
\footnotesize
\item whether stress controlled or strain controlled
\item strain/stress levels
\item RT or higher temperature
\item R values and  cycle waveform
\end{enumerate}

\item a) First cycle slip analysis with slip offsets or digital image correlation to determine initiation of slip b) slip analysis after quarter life or half-life to determine development of damage c) fracture initiation
\item data correlation with fatigue life
\end{enumerate}

\end{frame}


\begin{frame}{Timeline}

\begin{figure}
\includegraphics[width=\textwidth]{"images/Gnatt"}
\end{figure}

\end{frame}


\begin{frame}{Syllabus}
\textbf{Mechanical Behavior of materials}. Structure of materials, elastic \& plastic properties of materials, plastic flow in single and polycrystals, strengthening mechanisms, thermally activated deformation \\
\textbf{Defects in solids}. Point, line, area and volume defects, Dislocation-solute interaction \\
\textbf{Techniques for microstructural characterization}. Scanning Electron Microscope \\
\textbf{Texture}. X-Ray Texture, EBSD, Pole Figure, Orientation Distribution Function
\textbf{Physical Metallurgy of Titanium alloys}. Basic properties, crystal structure, deformation modes, phase diagrams, phase transformations, alloy classification, basic hardening mechanisms, effect of processing and composition on microstructure and mechanical properties, high temperature titanium alloys.
\end{frame}


\iffalse
\begin{frame}{Notations}

\begin{ganttchart}{1}{12}
  \gantttitle{2011}{12} \\
  \gantttitlelist{1,...,12}{1} \\
  \ganttgroup{Group 1}{1}{7} \\
  \ganttbar{Task 1}{1}{2} \\
  \ganttlinkedbar{Task 2}{3}{7} \ganttnewline
  \ganttmilestone{Milestone}{7} \ganttnewline
  \ganttbar{Final Task}{8}{12}
  \ganttlink{elem2}{elem3}
  \ganttlink{elem3}{elem4}
\end{ganttchart}



\end{frame}

\begin{frame}{Notations}

\begin{ganttchart}[
    y unit title=0.4cm,
    y unit chart=0.5cm,
    vgrid,
    time slot format=isodate-yearmonth,
    compress calendar,
    title/.append style={draw=none, fill=barblue},
    title label font=\sffamily\bfseries\color{white},
    title label node/.append style={below=-1.6ex},
    title left shift=.05,
    title right shift=-.05,
    title height=1,
    bar/.append style={draw=none, fill=groupblue},
    bar height=.6,
    bar label font=\normalsize\color{black!50},
    group right shift=0,
    group top shift=.6,
    group height=.3,
    group peaks height=.2,
    bar incomplete/.append style={fill=green}
   ]{2010-09}{2011-12}
   \gantttitlecalendar{year}\\
   \ganttbar[
    progress=100,
    bar progress label font=\small\color{barblue},
    bar progress label node/.append style={right=4pt},
    bar label font=\normalsize\color{barblue},
    name=pp
   ]{Preliminary Project}{2010-09}{2010-12} \\
\ganttset{progress label text={}, link/.style={black, -to}}
\ganttgroup{Objective 1}{2011-01}{2011-12} \\
\ganttbar[progress=4, name=T1A]{Task A}{2011-01}{2011-06} \\
\ganttlinkedbar[progress=0]{Task B}{2011-07}{2011-12} \\
\ganttgroup{Objective 2}{2011-01}{2011-12} \\
\ganttbar[progress=15, name=T2A]{Task A}{2011-01}{2011-09} \\
\ganttlinkedbar[progress=0]{Task B}{2011-10}{2011-12} \\
\ganttgroup{Objective 3}{2011-05}{2011-08} \\
  \ganttbar[progress=0]{Task A}{2011-05}{2011-08}
  \ganttset{link/.style={green}}
  \ganttlink[link mid=.4]{pp}{T1A}
  \ganttlink[link mid=.159]{pp}{T2A}
\end{ganttchart}


\end{frame}

\fi

%%%%%%%%%%%%%%%%%%%%%%% Thank You %%%%%%%%%%%%%%%%%%%%%%%%%%%

\begin{frame}[standout]
  Thank You
\end{frame}

   \bibliography{references}
  \bibliographystyle{abbrv}


\end{document}
