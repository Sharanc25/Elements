\documentclass[10pt]{beamer}

\usetheme[progressbar=frametitle]{metropolis}
\usepackage{appendixnumberbeamer}
\usepackage{booktabs}
\usepackage[scale=2]{ccicons}
\usepackage{xspace}
\newcommand{\themename}{\textbf{\textsc{metropolis}}\xspace}
\usepackage{amsmath}
\usepackage{graphicx}
\usepackage{comprehensive_preamble}
\usepackage[normalem]{ulem} % For text strikethrough
\usepackage{subcaption}
\usepackage{caption}
\captionsetup[subfigure]{labelformat=empty,position=b} % Removes Figure 1: from figures
\captionsetup[figure]{labelformat=empty}

\title{Primary Alpha, Transformed Beta and Low Cycle Fatigue of Titanium 6246 and 6242 Alloy}
\subtitle{Comprehensive Exam}
\date{\today}
\date{}
\author{Sharan Chandran}
\institute{Indian Institute of Science}
% \titlegraphic{\hfill\includegraphics[height=1.5cm]{logo.pdf}}

\begin{document}

\metroset{block=fill}

\maketitle

\begin{frame}[fragile]{Statement of work}

\begin{enumerate}
\item 3 heat treatments in alpha + beta and beta processed condition to generate different statistics of microstructure and microtexture.
\item Large and local area EBSD to generate
\begin{enumerate}
\footnotesize
\item adequate statistical description of primary alpha grains in terms of alpha/alpha grain contact
\item the statistics of primary grains that are BOR related to the surrounding beta
\item secondary alpha texture and its correlation with primary alpha texture
\item statistics of colony size and basket weave group size  for different heat treatments
\end{enumerate}
\item Generation of LCF data: 
\begin{enumerate}
\color{red}
\footnotesize
\item whether stress controlled or strain controlled
\item strain/stress levels
\item RT or higher temperature
\item R values and  cycle waveform
\end{enumerate}

\item a) First cycle slip analysis with slip offsets or digital image correlation to determine initiation of slip b) slip analysis after quarter life or half-life to determine development of damage c) fracture initiation
\item data correlation with fatigue life
\end{enumerate}

\end{frame}


\begin{frame}{Things to do before Comprehensive Exam}
\begin{enumerate}
\item Quantitative Metallography
\begin{enumerate}
\item $\alpha$ phase volume fraction
\item $\alpha$ phase grain size
\end{enumerate}
\item \sout{Heat Treatment - TBD. Maintain same $\alpha$ phase fraction (as received) with different $\alpha$ grain size.}
\item Tensile test for as received specimens - Ti-6242
\item Initial EBSD Texture
\item \color{red} Fatigue Test - For Ti-6242 at 0.95, 0.85, 0.8 YS
\end{enumerate}
\end{frame}

\begin{frame}{Initial Microstructure}
\begin{figure}[H]
    \centering
    \begin{subfigure}{0.45\textwidth}
        \includegraphics[width=\textwidth]{\HeatTreatment{"Ti6246-1.1-Top (500x)"}}
        \caption{Ti-6246}
        \label{fig:a-As-Received-micro}
    \end{subfigure}
    ~
    \begin{subfigure}{0.45\textwidth}
        \includegraphics[width=\textwidth]{\HeatTreatment{Ti6242-1.2-Top-5(500x).jpg}}
        \caption{Ti-6242}
        \label{fig:Ti-6242 Surface}
    \end{subfigure}
  
    \caption{As Received Microstructure; Magnification - 500x}
    \label{fig:As-Received-micro}
\end{figure}
\end{frame}

\begin{figure}[H]
    \centering
        \includegraphics[width=0.75\textwidth]{\HeatTreatment{Voulme_Fraction.eps}}
        \caption{Volume Fraction at different temperatures for Ti-6242}
    \label{fig:Volume Fraction at different temperatures for Ti-6242}
\end{figure}


%%%%%%%%%%%%%%%%%%%%%%% Thank You %%%%%%%%%%%%%%%%%%%%%%%%%%%

\begin{frame}[standout]
  Thank You
\end{frame}

\end{document}
