\documentclass[10pt]{beamer}

\usetheme[progressbar=frametitle]{metropolis}
\usepackage{appendixnumberbeamer}

\usepackage{booktabs}
\usepackage[scale=2]{ccicons}

\usepackage{pgfplots}
\usepgfplotslibrary{dateplot}

\usepackage{xspace}
\newcommand{\themename}{\textbf{\textsc{metropolis}}\xspace}

\usepackage{amsmath}
\usepackage{graphicx}

\usepackage{pgfgantt} % To make Gantt Charts

\title{Primary Alpha, Transformed Beta and Low Cycle Fatigue of Titanium 6246 and 6242}
\subtitle{Comprehensive Exam}
\date{\today}
\date{}
\author{Sharan Chandran}
\institute{Indian Institute of Science}
% \titlegraphic{\hfill\includegraphics[height=1.5cm]{logo.pdf}}

\begin{document}

\maketitle

\begin{frame}{Table of contents}
  \setbeamertemplate{section in toc}[sections numbered]
  \tableofcontents[hideallsubsections]
\end{frame}

\section{Objectives}

\begin{frame}[fragile]{Objectives}
\begin{itemize}
\item 3 heat treatments in alpha +beta and beta processed condition to generate different statistics of microstructure and microtexture.

\item Large and local area EBSD to generate an a) adequate statistical description of primary alpha grains in terms of alpha/alpha grain contact b) the statistics of primary grains that are BOR related to the surrounding beta c) secondary alpha texture and its correlation with primary alpha texture  d) statistics of colony size and basket weave group size  for different heat treatments

\item Generation of LCF data: a) whether stress controlled or strain controlled b) strain/stress levels c) RT or higher temperature d) R values and  cycle waveform 

\item a) First cycle slip analysis with slip offsets or digital image correlation to determine initiation of slip b) slip analysis after quarter life or half-life to determine development of damage c) fracture initiation

\item data correlation with fatigue life

\end{itemize}
\end{frame}

\metroset{block=fill}

\begin{frame}{Notations}

\begin{ganttchart}[
    canvas/.append style={fill=none, draw=black!5, line width=.75pt},
    hgrid style/.style={draw=black!5, line width=.75pt},
    vgrid={*1{draw=black!5, line width=.75pt}},
    today=7,
    today rule/.style={
      draw=black!64,
      dash pattern=on 3.5pt off 4.5pt,
      line width=1.5pt
    },
    today label font=\small\bfseries,
    title/.style={draw=none, fill=none},
    title label font=\bfseries\footnotesize,
    title label node/.append style={below=7pt},
    include title in canvas=false,
    bar label font=\mdseries\small\color{black!70},
    bar label node/.append style={left=2cm},
    bar/.append style={draw=none, fill=black!63},
    bar incomplete/.append style={fill=barblue},
    bar progress label font=\mdseries\footnotesize\color{black!70},
    group incomplete/.append style={fill=groupblue},
    group left shift=0,
    group right shift=0,
    group height=.5,
    group peaks tip position=0,
    group label node/.append style={left=.6cm},
    group progress label font=\bfseries\small,
    link/.style={-latex, line width=1.5pt, linkred},
    link label font=\scriptsize\bfseries,
    link label node/.append style={below left=-2pt and 0pt}
  ]{1}{13}
  \gantttitle[
    title label node/.append style={below left=7pt and -3pt}
  ]{WEEKS:\quad1}{1}
  \gantttitlelist{2,...,13}{1} \\
  \ganttgroup[progress=57]{WBS 1 Summary Element 1}{1}{10} \\
  \ganttbar[
    progress=75,
    name=WBS1A
  ]{\textbf{WBS 1.1} Activity A}{1}{8} \\
  \ganttbar[
    progress=67,
    name=WBS1B
  ]{\textbf{WBS 1.2} Activity B}{1}{3} \\
  \ganttbar[
    progress=50,
    name=WBS1C
  ]{\textbf{WBS 1.3} Activity C}{4}{10} \\
  \ganttbar[
    progress=0,
    name=WBS1D
  ]{\textbf{WBS 1.4} Activity D}{4}{10} \\[grid]
  \ganttgroup[progress=0]{WBS 2 Summary Element 2}{4}{10} \\
  \ganttbar[progress=0]{\textbf{WBS 2.1} Activity E}{4}{5} \\
  \ganttbar[progress=0]{\textbf{WBS 2.2} Activity F}{6}{8} \\
  \ganttbar[progress=0]{\textbf{WBS 2.3} Activity G}{9}{10}
  \ganttlink[link type=s-s]{WBS1A}{WBS1B}
  \ganttlink[link type=f-s]{WBS1B}{WBS1C}
  \ganttlink[
    link type=f-f,
    link label node/.append style=left
  ]{WBS1C}{WBS1D}
\end{ganttchart}

\end{frame}



{%
\setbeamertemplate{frame footer}{Slide 14 - https://www.brainshark.com/malvern/vu?pi=577133879\& text=M021507\& r3f1=}
\begin{frame}[fragile]{Theoretical Density}

\begin{equation*}
Theoretical Density = \dfrac{\text{MW * No.of Molecules per unit volume}}{\text{Volume of unit cell * Avogadro number}}
\end{equation*}   
    
\end{frame}
}
%%%%%%%%%%%%%%%%%%%%%%%%%%%%%%%%%%%%%%%%%%%%%%%%%%%%%%%%%%%%%%%%%%%%%%%%%%%%%%%%%%%%
{%
\setbeamertemplate{frame footer}{Slide 14 - https://www.brainshark.com/malvern/vu?pi=577133879\& text=M021507\& r3f1=}
\begin{frame}[fragile]{Volume Fraction}


  \begin{columns}[T,onlytextwidth]
 
    \column{0.49\textwidth}
\begin{block}{Areal Fraction}
\begin{equation*}
V_{v} = \dfrac{\sum A_{A}}{A_{T}}
\end{equation*} 
\end{block}
 
 \column{0.49\textwidth}

\begin{block}{Error}
\begin{equation*}
E_{A}^{2} = \dfrac{1}{N}\left[ 1+ \left( \dfrac{\sigma}{\bar{A}} \right)^{2} \right]
\end{equation*} 
\end{block}

\end{columns}

  \begin{columns}[T,onlytextwidth]
    \column{0.49\textwidth}
\begin{block}{Lineal Fraction}
\begin{equation*}
L_{v} = \dfrac{\sum A_{A}}{A_{T}}
\end{equation*} 
\end{block}
 
 \column{0.49\textwidth}

\begin{block}{Error}
\small
\begin{equation*}
E_{L}^{2} = \dfrac{1}{N}(1-V_{V})^{2}\left[ \left( \dfrac{\sigma^{\alpha}_{L}}{\bar{L_{\alpha}}} \right)^{2} + \left( \dfrac{\sigma^{\beta}_{L}}{\bar{L_{\beta}}} \right)^{2} \right]
\end{equation*}

\end{block}

\end{columns}

  \begin{columns}[T,onlytextwidth]
    \column{0.49\textwidth}
\begin{block}{Point Fraction}
\begin{equation*}
P_{v} = \dfrac{\sum A_{A}}{A_{T}}
\end{equation*} 
\end{block}
 
 \column{0.49\textwidth}

\begin{block}{Error}
\begin{equation*}
E_{P}^{2} = \dfrac{1}{P}
\end{equation*} 
\end{block}

\end{columns}


 
    
\end{frame}
}
%%%%%%%%%%%%%%%%%%%%%%%%%%%%%%%%%%%%%%%%%%%%%%%%%%%%%%%%%%%%%%%%%%%%%%%%%%%%%%%%%%%%
\section{Titleformats}



%%%%%%%%%%%%%%%%%%%%%%%%%%%%%%%%%%%%%%%%%%%%%%%%%%%%%%%%%%%%%%%%%%%%%%%%%%%%%%%%%%%%
{
    \metroset{titleformat frame=smallcaps}
\begin{frame}{Small caps}

\end{frame}
}

%%%%%%%%%%%%%%%%%%%%%%%%%%%%%%%%%%%%%%%%%%%%%%%%%%%%%%%%%%%%%%%%%%%%%%%%%%%%%%%%%%%%
{
\metroset{titleformat frame=allsmallcaps}
\begin{frame}{All small caps}

\end{frame}
}
%%%%%%%%%%%%%%%%%%%%%%%%%%%%%%%%%%%%%%%%%%%%%%%%%%%%%%%%%%%%%%%%%%%%%%%%%%%%%%%%%%%%
{
\metroset{titleformat frame=allcaps}
\begin{frame}{All caps}

\end{frame}
}

%%%%%%%%%%%%%%%%%%%%%%%%%%%%%%%%%%%%%%%%%%%%%%%%%%%%%%%%%%%%%%%%%%%%%%%%%%%%%%%%%%%%
\begin{frame}{Blocks}
 

      \begin{block}{Default}
        Block content.
      \end{block}

      \begin{alertblock}{Alert}
        Block content.
      \end{alertblock}

      \begin{exampleblock}{Example}
        Block content.
      \end{exampleblock}


\end{frame}

%%%%%%%%%%%%%%%%%%%%%%%%%%%%%%%%%%%%%%%%%%%%%%%%%%%%%%%%%%%%%%%%%%%%%%%%%%%%%%%%%%%%
\begin{frame}{References}
  Some references to showcase [allowframebreaks] \cite{knuth92,ConcreteMath,Simpson,Er01,greenwade93}
\end{frame}
%%%%%%%%%%%%%%%%%%%%%%%%%%%%%%%%%%%%%%%%%%%%%%%%%%%%%%%%%%%%%%%%%%%%%%%%%%%%%%%%%%%%
\section{Conclusion}

\begin{frame}{Summary}

\end{frame}

%%%%%%%%%%%%%%%%%%%%%%%%%%%%%%%%%%%%%%%%%%%%%%%%%%%%%%%%%%%%%%%%%%%%%%%%%%%%%%%%%%%%
{\setbeamercolor{palette primary}{fg=black, bg=yellow}
\begin{frame}[standout]
  Questions?
\end{frame}
}

\appendix
%%%%%%%%%%%%%%%%%%%%%%%%%%%%%%%%%%%%%%%%%%%%%%%%%%%%%%%%%%%%%%%%%%%%%%%%%%%%%%%%%%%%
\begin{frame}[fragile]{Backup slides}
Test
\end{frame}

%%%%%%%%%%%%%%%%%%%%%%%%%%%%%%%%%%%%%%%%%%%%%%%%%%%%%%%%%%%%%%%%%%%%%%%%%%%%%%%%%%%%
\begin{frame}[allowframebreaks]{References}

  \bibliography{references}
  \bibliographystyle{abbrv}

\end{frame}

\end{document}
