\documentclass[10pt]{beamer}

\usetheme[progressbar=frametitle]{metropolis}
\usepackage{appendixnumberbeamer}
\usepackage{booktabs}
\usepackage[scale=2]{ccicons}
\usepackage{xspace}
\newcommand{\themename}{\textbf{\textsc{metropolis}}\xspace}
\usepackage{amsmath}
\usepackage{graphicx}
\usepackage{comprehensive_preamble}
\usepackage[normalem]{ulem} % For text strikethrough
\usepackage{subcaption}
\usepackage{caption}
\captionsetup[subfigure]{labelformat=empty,position=b} % Removes Figure 1: from figures
\captionsetup[figure]{labelformat=empty}

\title{Low Cycle Fatigue Studies on Titanium 6242 and 6246 Alloy}
\subtitle{Comprehensive Exam}
\date{\today}
\date{}
\author{Sharan Chandran}
\institute{Indian Institute of Science}
% \titlegraphic{\hfill\includegraphics[height=1.5cm]{logo.pdf}}

\begin{document}

\metroset{block=fill}

\maketitle


\iffalse
\begin{frame}{Things to do before Comprehensive Exam}
\begin{enumerate}
\item Quantitative Metallography
\begin{enumerate}
\item $\alpha$ phase volume fraction
\item $\alpha$ phase grain size
\end{enumerate}
\item \sout{Heat Treatment - TBD. Maintain same $\alpha$ phase fraction (as received) with different $\alpha$ grain size.}
\item Tensile test for as received specimens - Ti-6242
\item Initial EBSD Texture
\item \color{red} Fatigue Test - For Ti-6242 at 0.95, 0.85, 0.8 YS
\end{enumerate}
\end{frame}
\fi

%%%%%%%%%%%%%%% Introduction - Physical Metallurgy of Titanium
{%
\setbeamertemplate{frame footer}{Titanium, Gerd L, James C.W.}
\begin{frame}[fragile]{Physical Metallurgy of Titanium}

\begin{figure}[H]
    \centering
        \includegraphics[width=0.90\textwidth]{images/Crystal Structure.jpg}
        %\caption{S-N Curve for Ti-6242}
\end{figure}

Pure titanium undergoes allotrophic transformation at 882 \degC

%Vanadium replaced by Mo.

%Moly diffusivity is lower than V in Ti.

%Titanium 6246 - Compressor discs and fan blades

%Effect of Sn and Zr

\end{frame}
}

%%%%%%%%%%%%%%% Introduction - Physical Metallurgy of Titanium
{%
\setbeamertemplate{frame footer}{Titanium, Gerd L, James C.W.}

\begin{frame}[fragile]{Physical Metallurgy of Titanium}

\begin{figure}[H]
    \centering
        \includegraphics[width=0.70\textwidth]{images/Phase Diagram.jpg}
        %\caption{S-N Curve for Ti-6242}
        
        
\end{figure}

\end{frame}
}

%%%%%%%%%%%%%%% Introduction - Physical Metallurgy of Titanium
{%
\setbeamertemplate{frame footer}{Titanium, Gerd L, James C.W.}

\begin{frame}[fragile]{Alloying of Titanium}

\begin{table}[]
\centering
\begin{tabular}{@{}cccccc@{}}
\toprule
 & \textbf{Al} & \textbf{Sn} & \textbf{Zr} & \textbf{Mo} & \textbf{Si} \\ \midrule
\textbf{Ti-6242} & 6 & 2 & 4 & 2 & 0.1 \\
\textbf{Ti-6246} & 6 & 2 & 4 & 6 & - \\ \bottomrule
\end{tabular}
\caption{My caption}
\end{table}
Sn - Beta Stabilizer; Solid Solution Strengthening
\end{frame}
}

%%%%%%%%% Ti6242 and Ti6246 Microstructure
{%
\setbeamertemplate{frame footer}{Titanium, Gerd L, James C.W.;http://www.france-metallurgie.com/}
\begin{frame}[fragile]{Applications of Ti-6242 and Ti-6246}

\begin{figure}[H]
    \centering
    \begin{subfigure}{0.45\textwidth}
        \includegraphics[width=\textwidth]{images/Impeller Ti-6242.jpg}
        \caption{Impeller used in a small engine, Ti-6242}
        \end{subfigure}
    ~
    \begin{subfigure}{0.45\textwidth}
        \includegraphics[width=\textwidth]{images/aeroengine Ti-6246.jpg}
        \caption{Ti-6246 in Aircraft engine}
    \end{subfigure}
  
    %\caption{}
    
\end{figure}

\end{frame}
}

%%%%%%%%%%%%%%% Introduction - Fatigue Test
{%
\setbeamertemplate{frame footer}{Sinha V., Mills M. J., Williams J. C.: Lightweight Alloys for Aerospace Application, TMS, Warrendale, USA, (2001) p. 194}
\begin{frame}[fragile]{Fatigue in Titanium Alloys}

\begin{figure}[H]
    \centering
    \begin{subfigure}{0.30\textwidth}
        \includegraphics[width=\textwidth]{images/alpha-beta-forged-high-microtexture.jpg}
        \caption{\tiny $\alpha+\beta$ forged (high microtexture)}
        \label{fig:Ti-6242 Surface}
    \end{subfigure}
    \\
    \begin{subfigure}{0.30\textwidth}
        \includegraphics[width=\textwidth]{images/alpha-beta-forged-high-microtexture(1980)}
        \caption{\tiny  $\alpha+\beta$ forged (high microtexture - 1980)}
        \label{fig:Ti-6242 Surface}
    \end{subfigure}
    ~
    \begin{subfigure}{0.30\textwidth}
        \includegraphics[width=\textwidth]{images/beta-forged.jpg}
        \caption{\tiny $\beta$ forged}
        \label{fig:Ti-6242 Surface}
    \end{subfigure}        
  
    %\caption{As Received Microstructure; Magnification - 500x}
    
\end{figure}

\end{frame}
}

%%%%%%%%%%%%%%% Intorduction - Literature Survey
{%
\setbeamertemplate{frame footer}{Arunima et al}
\begin{frame}[fragile]{Literature Survey}

Alpha + Beta Heat Treated Structures \\
In a recent publications [1, 2] we have shown that primary, equiaxed alpha can exist both the BOR and in non- BOR relationship with surrounding beta grains in an alpha+beta processed microstructure. We also note that such primary alpha grains need not be completely isolated from other such grains but can remain in contact with each other in local areas. A simple schematic description of these possibilities is shown in Figure1.This aspect of the crystallographic relationship with of equiaxed, primary alpha with surrounding beta grains or alpha/alpha grain contact has not been examined in any great detail in earlier work, particularly since it has been commonly believed that such recrystallized equiaxed alpha with not share a BOR with surrounding beta grains.

\end{frame}
}
%%%%%%%%%%%%%%% Objectives

\begin{frame}[fragile]{Objectives}

\begin{enumerate}
\item Study the effect of microstructure and microtexture on low cycle fatigue of Ti-6242 and Ti-6246 alloys.
\item Secondary alpha texture and its correlation with primary alpha texture.
\item First cycle slip analysis to determine initiation of slip.
\item Slip analysis after quarter-life or half-life to determine development of damage.
\item Fracture analysis.
\end{enumerate}

\end{frame}

%%%%%%%%% Ti6242 and Ti6246 Microstructure
{%
\setbeamertemplate{frame footer}{Arunima et al}
\begin{frame}[fragile]{Objectives}



\end{frame}
}

%%%%%%%%% Initial Microstructure
\begin{frame}{Initial Microstructure}

Ti-6Al-2Sn-4Zr-2Mo-0.1Si and Ti-6Al-2Sn-4Zr-6Mo

\begin{figure}[H]
    \centering
    \begin{subfigure}{0.45\textwidth}
        \includegraphics[width=\textwidth]{\HeatTreatment{Ti6242-1.2-Top-5(500x).jpg}}
        \caption{Ti-6242}
        \end{subfigure}
    ~
    \begin{subfigure}{0.45\textwidth}
        \includegraphics[width=\textwidth]{\HeatTreatment{Ti6246-1.1-Top (500x)}}
        \caption{Ti-6246}
    \end{subfigure}
  
    \caption{Optical Micrograph of as Received Microstructure; Magnification - 500x}
    
\end{figure}
\end{frame}

%%%%%%%%%%%%%%% Objectives

\begin{frame}[fragile]{Effect of equiaxed alpha Volume fraction on Fatigue life - Ti-6242}
Beta Transus - 1005 $\pm$5 \degC
Initial Volume Fraction 


\end{frame}


%%%%%%%%% Volume Fraction at different temperatures for Ti-6242 - 1

\begin{frame}{Equiaxed Aplha Volume Fraction at different temperatures for Ti-6242}
\begin{figure}[H]
    \centering
    \begin{subfigure}{0.25\textwidth}
        \includegraphics[width=\textwidth]{images/Ti6242-1.1.5-CS-HT701-5-(200x).png}
        \caption{701\degC ()}
        \label{fig:Ti-6242 HT700}
    \end{subfigure}    
    ~
    \begin{subfigure}{0.25\textwidth}
        \includegraphics[width=\textwidth]{images/Ti6242-1.2.1-Top-HT765(2.5)-6-(200x).png}
        \caption{765\degC}
        \label{fig:Ti-6242 HT752}
    \end{subfigure}   
   \\
    \begin{subfigure}{0.25\textwidth}
        \includegraphics[width=\textwidth]{images/Ti6242-1.1.3-Top-(HT815)-5-(200x).png}
        \caption{815\degC}
        \label{fig:Ti-6242 HT815}
    \end{subfigure}
    ~
    \begin{subfigure}{0.25\textwidth}
        \includegraphics[width=\textwidth]{images/Ti6242-1.1.2-Top-HT863-5-(200x).png}
        \caption{863\degC}
        \label{fig:Ti-6242 HT863}
    \end{subfigure}
    ~
    \begin{subfigure}{0.25\textwidth}
        \includegraphics[width=\textwidth]{images/Ti6242-1.1-Top-HT900(2)-13-(200x).png}
        \caption{920\degC}
        \label{fig:Ti-6242 HT920}
    \end{subfigure}        
   
    \caption{Microstructure of Ti-6242 for different heat treatments; Marker - 50$\mu$m}
    
\end{figure}
\end{frame}



%%%%%%%%% Volume Fraction at different temperatures for Ti-6242 - 2

\begin{frame}{Equiaxed Aplha Volume Fraction at different temperatures for Ti-6242}
\begin{figure}[H]
    \centering
        \includegraphics[width=0.75\textwidth]{\HeatTreatment{Voulme_Fraction.eps}}
        \caption{Volume Fraction at different temperatures for Ti-6242}
\end{figure}
\end{frame}

%%%%%%%%% Tensile Test for Ti-6242 @0.0333 s$^{-1}$
\begin{frame}{Tensile Test for Ti-6242}

\begin{figure}[H]
    \centering
        \includegraphics[width=0.80\textwidth]{\TensileTest{Ti6242-1.5-TS-Graph.eps}}
    %\caption{Tensile Test of as received Ti-6242 @0.0333 s$^{-1}$}
    \\
	Strain Rate - @0.0333 s$^{-1}$    
	\\
    Room Temperature
    
\end{figure}

\end{frame}

%%%%%%%%%%%%%%% Introduction - Fatigue Test
{%
\setbeamertemplate{frame footer}{}
\begin{frame}[fragile]{Fatigue Test Schematic}

\begin{figure}[H]
    \centering
    \begin{subfigure}{0.40\textwidth}
        \includegraphics[width=\textwidth]{images/Cyclic Fatigue Schematic.png}
        \caption{Cyclic Fatigue}
    \end{subfigure}
    \\
    \begin{subfigure}{0.60\textwidth}
        \includegraphics[width=\textwidth]{images/Dwell Fatigue Schematic.png}
        \caption{Dwell Fatigue}
    \end{subfigure}    
\end{figure}

\end{frame}
}

%%%%%%%%% Fatigue Test for Ti-6242
\begin{frame}{Stress Controlled Fatigue Test for As Received Ti-6242}
\begin{figure}[H]
    \centering
        \includegraphics[width=0.70\textwidth]{\FatigueTest{S-N_Curve.eps}}
        %\caption{S-N Curve for Ti-6242}
\end{figure}

\end{frame}
%%%%%%%%%%%%%%%%%%%%%%% Thank You %%%%%%%%%%%%%%%%%%%%%%%%%%%

\begin{frame}[standout]
  Thank You
\end{frame}

%%%%%%%%% DSC Curve for Ti-6242
\begin{frame}{Measuring Beta Transus for Ti-6242}
\begin{figure}[H]
    \centering
    \begin{subfigure}{0.48\textwidth}
        \includegraphics[width=\textwidth]{\HeatTreatment{Ti6242-Heat-Graph.eps}}
        \caption{Heating Cycle}
        \label{fig:Ti-6242 Threshold}
    \end{subfigure}
    ~
    \begin{subfigure}{0.48\textwidth}
        \includegraphics[width=\textwidth]{\HeatTreatment{Ti6242-Cool-Graph.eps}}
        \caption{Cooling Cycle}
        \label{fig:Ti-6242 HT700}
    \end{subfigure}    
   
    \caption{DSC Curve for Ti-6242 @10 Kpm}
  
\end{figure}
\end{frame}

\end{document}
