\documentclass[10pt]{beamer}

\usetheme[progressbar=frametitle]{metropolis}
\usepackage{appendixnumberbeamer}

\usepackage{booktabs}
\usepackage[scale=2]{ccicons}

\usepackage{pgfplots}
\usepgfplotslibrary{dateplot}

\usepackage{xspace}
\newcommand{\themename}{\textbf{\textsc{metropolis}}\xspace}

\usepackage{amsmath}
\usepackage{graphicx}

\usepackage{pgfgantt} % To make Gantt Charts

\usepackage[normalem]{ulem} % For text strikethrough

\title{Primary Alpha, Transformed Beta and Low Cycle Fatigue of Titanium 6246 and 6242 Alloy}
\subtitle{Comprehensive Exam}
\date{\today}
\date{}
\author{Sharan Chandran}
\institute{Indian Institute of Science}
% \titlegraphic{\hfill\includegraphics[height=1.5cm]{logo.pdf}}

\begin{document}

\metroset{block=fill}

%%%%%%%%%%%%%% For Gantt Chart %%%%%%%%%%%%%%%%%%%
\definecolor{barblue}{RGB}{153,204,254}
\definecolor{groupblue}{RGB}{51,102,254}
\definecolor{linkred}{RGB}{165,0,33}
\renewcommand\sfdefault{phv}
\renewcommand\mddefault{mc}
\renewcommand\bfdefault{bc}
\setganttlinklabel{s-s}{START-TO-START}
\setganttlinklabel{f-s}{FINISH-TO-START}
\setganttlinklabel{f-f}{FINISH-TO-FINISH}
\sffamily

\maketitle

\begin{frame}{Table of contents}
  \setbeamertemplate{section in toc}[sections numbered]
  \tableofcontents[hideallsubsections]
\end{frame}

\section{Objectives}

\begin{frame}[fragile]{Statement of work}

\begin{enumerate}
\item 3 heat treatments in alpha + beta and beta processed condition to generate different statistics of microstructure and microtexture.
\item Large and local area EBSD to generate
\begin{enumerate}
\footnotesize
\item adequate statistical description of primary alpha grains in terms of alpha/alpha grain contact
\item the statistics of primary grains that are BOR related to the surrounding beta
\item secondary alpha texture and its correlation with primary alpha texture
\item statistics of colony size and basket weave group size  for different heat treatments
\end{enumerate}
\item Generation of LCF data: 
\begin{enumerate}
\color{red}
\footnotesize
\item whether stress controlled or strain controlled
\item strain/stress levels
\item RT or higher temperature
\item R values and  cycle waveform
\end{enumerate}

\item a) First cycle slip analysis with slip offsets or digital image correlation to determine initiation of slip b) slip analysis after quarter life or half-life to determine development of damage c) fracture initiation
\item data correlation with fatigue life
\end{enumerate}

\end{frame}


\begin{frame}{Timeline}

\begin{figure}
\includegraphics[width=\textwidth]{"images/Gnatt"}
\end{figure}

\end{frame}


\begin{frame}{Syllabus}
\textbf{Mechanical Behavior of materials} Deformation of single and poly crystals. Temperature and strain rate effects in plastic flow - strain hardening. Tensile, Fatigue and fracture. \\
\textbf{Structure and characterization of materials.} Electron diffraction and Electron microscopy. Resolution and Rayleigh criterion, electron optics, electron guns and lenses, probe diameter and probe current, electron-specimen interactions, interaction volume. Principles of scanning electron microscopy, imaging modes and detectors. \\
\textbf{Texture.} Concepts of texture in materials, their representation by pole figure and orientation distribution functions. Texture measurement by different techniques. Origin and development of texture during
material processing stages: solidification, deformation, annealing, phase transformation. Influence of texture on mechanical and physical properties. 

\end{frame}

\begin{frame}{Things to do before Comprehensive Exam}
\begin{enumerate}
\item Quantitative Metallography
\begin{enumerate}
\item $\alpha$ phase volume fraction
\item $\alpha$ phase grain size
\end{enumerate}
\item \sout{Heat Treatment - TBD. Maintain same $\alpha$ phase fraction (as received) with different $\alpha$ grain size.}
\item Tensile test for as received specimens - Ti-6242
\item Initial EBSD Texture
\item \color{red} Fatigue Test - For Ti-6242 at 0.95, 0.85, 0.8 YS
\end{enumerate}
\end{frame}



%%%%%%%%%%%%%%%%%%%%%%% Thank You %%%%%%%%%%%%%%%%%%%%%%%%%%%

\begin{frame}[standout]
  Thank You
\end{frame}

   \bibliography{references}
  \bibliographystyle{abbrv}


\end{document}
