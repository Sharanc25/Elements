\documentclass[10pt]{beamer}

\usetheme[progressbar=frametitle]{metropolis}
\usepackage{appendixnumberbeamer}
\usepackage{booktabs}
\usepackage[scale=2]{ccicons}
\usepackage{xspace}
\newcommand{\themename}{\textbf{\textsc{metropolis}}\xspace}
\usepackage{amsmath}
\usepackage{graphicx}
\usepackage{comprehensive_preamble}
\usepackage[normalem]{ulem} % For text strikethrough
\usepackage{subcaption}
\usepackage{caption}
\captionsetup[subfigure]{labelformat=empty,position=b} % Removes Figure 1: from figures
\captionsetup[figure]{labelformat=empty}

\title{Primary Alpha, Transformed Beta and Low Cycle Fatigue of Titanium 6246 and 6242 Alloy}
\subtitle{Comprehensive Exam}
\date{\today}
\date{}
\author{Sharan Chandran}
\institute{Indian Institute of Science}
% \titlegraphic{\hfill\includegraphics[height=1.5cm]{logo.pdf}}

\begin{document}

\metroset{block=fill}

\maketitle

\begin{frame}[fragile]{Statement of work}

\begin{enumerate}
\item Study the effect of microstructure and microtexture on low cycle fatigue of Ti-6242 and Ti-6246 alloys.
\item Secondary alpha texture and its correlation with primary alpha texture.
\item First cycle slip analysis to determine initiation of slip.
\item Slip analysis after quarter-life or half-life to determine development of damage.
\item Fracture analysis.
\end{enumerate}

\end{frame}

\iffalse
\begin{frame}{Things to do before Comprehensive Exam}
\begin{enumerate}
\item Quantitative Metallography
\begin{enumerate}
\item $\alpha$ phase volume fraction
\item $\alpha$ phase grain size
\end{enumerate}
\item \sout{Heat Treatment - TBD. Maintain same $\alpha$ phase fraction (as received) with different $\alpha$ grain size.}
\item Tensile test for as received specimens - Ti-6242
\item Initial EBSD Texture
\item \color{red} Fatigue Test - For Ti-6242 at 0.95, 0.85, 0.8 YS
\end{enumerate}
\end{frame}
\fi

%%%%%%%%% Initial Microstructure
\begin{frame}{Initial Microstructure}
\begin{figure}[H]
    \centering
    \begin{subfigure}{0.45\textwidth}
        \includegraphics[width=\textwidth]{\HeatTreatment{"Ti6246-1.1-Top (500x)"}}
        \caption{Ti-6246}
        \label{fig:a-As-Received-micro}
    \end{subfigure}
    ~
    \begin{subfigure}{0.45\textwidth}
        \includegraphics[width=\textwidth]{\HeatTreatment{Ti6242-1.2-Top-5(500x).jpg}}
        \caption{Ti-6242}
        \label{fig:Ti-6242 Surface}
    \end{subfigure}
  
    \caption{As Received Microstructure; Magnification - 500x}
    
\end{figure}
\end{frame}

%%%%%%%%% DSC Curve for Ti-6242
\begin{frame}{DSC Curve for Ti-6242}
\begin{figure}[H]
    \centering
    \begin{subfigure}{0.48\textwidth}
        \includegraphics[width=\textwidth]{\HeatTreatment{Ti6242-Heat-Graph.eps}}
        \caption{Heating Cycle}
        \label{fig:Ti-6242 Threshold}
    \end{subfigure}
    ~
    \begin{subfigure}{0.48\textwidth}
        \includegraphics[width=\textwidth]{\HeatTreatment{Ti6242-Cool-Graph.eps}}
        \caption{Cooling Cycle}
        \label{fig:Ti-6242 HT700}
    \end{subfigure}    
   
    \caption{DSC Curve for Ti-6242 @10 Kpm}
  
\end{figure}
\end{frame}

%%%%%%%%% Volume Fraction at different temperatures for Ti-6242
\begin{frame}{Volume Fraction at different temperatures for Ti-6242}
\begin{figure}[H]
    \centering
        \includegraphics[width=0.75\textwidth]{\HeatTreatment{Voulme_Fraction.eps}}
        \caption{Volume Fraction at different temperatures for Ti-6242}
\end{figure}
\end{frame}

%%%%%%%%% Tensile Test for Ti-6242 @0.0333 s$^{-1}$
\begin{frame}{Tensile Test for Ti-6242}
\begin{figure}[H]
    \centering
        \includegraphics[width=0.90\textwidth]{\TensileTest{Ti6242-1.4-TS-Graph}}
    %\caption{Tensile Test of as received Ti-6242 @0.0333 s$^{-1}$}
\end{figure}
\end{frame}

%%%%%%%%% Fatigue Test for Ti-6242
\begin{frame}{Fatigue Test for Ti-6242}
\begin{figure}[H]
    \centering
        \includegraphics[width=0.90\textwidth]{\FatigueTest{S-N_Curve.eps}}
        %\caption{S-N Curve for Ti-6242}
\end{figure}
\end{frame}
%%%%%%%%%%%%%%%%%%%%%%% Thank You %%%%%%%%%%%%%%%%%%%%%%%%%%%

\begin{frame}[standout]
  Thank You
\end{frame}

\end{document}
