\chapter{Literature Review}
Titanium can exist in two allotropic forms: alpha (a hexagonal close-packed crystal structure) and beta (a body-cen- tered cubic structure) (Ref 7.1–7.4). In pure tita- nium, the alpha (a) phase is stable up to 880 °C (1620 °F), at which point it transforms to the beta (b) phase; the beta phase is stable from 880 °C (1620 °F) to the melting point. At room temperature, pure titanium consists of the alpha phase. However, the alloys can contain alpha, mixtures of alpha and beta, or beta phases, de- pending on the alloy content and conditions. Thus, the alloys are classified into these structural types: alpha ($\alpha$), alpha-beta ($\alpha$-$\beta$), and beta ($\beta$).


There are two major breakdowns in classifying the alloying elements. These are based on: A- whether or not the alloying elements are between the titanium atoms (interstitial) or replace titanium atoms (substitutional), and B-whether the alpha phase is entered preferentially (alpha stabilizing) or the beta phase is entered preferentially (beta stabilizing). The beta stabilizing elements generally are classified further, depending on whether or not there is a continuous series of solid solution between the alloying element and beta titanium (beta isomorphous), or the beta phase decomposes eutectoidally (beta eutectoid).


Ti-6246 is a α+β titanium alloy with nominal composition: Ti‒6Al‒2Sn‒4Zr‒6Mo (wt\%). In titanium alloys, the effect of β stabilizers is usually described in terms of their molybdenum equivalent ([Mo]eq = [Mo] + 0.2 [Ta] +0.28 [Nb] + 0.4 [W] + 0.67 [V] + 1.25 [Cr] + 1.25 [Ni] + 1.7 [Mn] + 1.7 [Co] + 2.5 [Fe]), and the [Mo]eq of Ti-6246 is calculated to be 6. Previously reported works on Ti‒Mo system alloys have shown that water quenching from β-phase field resulted in a matensitic transformation, and the martensite structure changed from hexagonal (α′) to orthorhombic (α″) at ~4 wt\% Mo [22]. The location of Ti-6246 in the pseudo-binary section through
a β isomorphous phase diagram[23] is shown in Fig. 9. It is shown that the group of α+β alloys has a
range in the phase diagram from the α/α+β phase boundary up to the intersection of the Ms-line
with room temperature, thus Ti-6246 transform martensitically upon fast cooling from the β-phase
field to room temperature. According to the present results, there exists no α″ phase in the XRD
patterns of the FZ, which indicates that the cooling rate in FZ during welding did not reach the
critical cooling rate of the martensitic transformation.