\chapter{Experimental Details}
\section{Titanium 6246 Alloy} 

\begin{figure}[H]
    \centering
    \begin{subfigure}{0.40\textwidth}
        \includegraphics[width=\textwidth]{\Photos{"Ti6246 [uncut]"}}
        \caption{Ti-6246 As Received; Surface}
        \label{fig:2a}
    \end{subfigure}
    ~
    \begin{subfigure}{0.40\textwidth}
        \includegraphics[width=\textwidth]{\Photos{"Ti6246 [uncut-cs]"}}
        \caption{Ti-6246 As Received; Cross-Section}
        \label{fig:2b}
    \end{subfigure}
    %\caption{}
    \label{fig:As-Received}
\end{figure}

\section{Titanium 6242 Alloy} 

\begin{figure}[H]
    \centering
        \includegraphics[width=0.50\textwidth]{\Photos{"Ti6242-1-Flat"}}
        \caption{Ti-6242 As Received; Surface}
    \label{fig:EDM-Cut}
\end{figure}


\subsection{EDM Cutting}
The samples were cut in EDM. To identify Surface and Cross-Section samples, an additional L-shaped groove was made on all the Cross-Section samples. 

\begin{figure}[H]
    \centering
        \includegraphics[width=\textwidth]{\Photos{"Ti6246 [cut-EDM]"}}
        \caption{Ti-6246 EDM Cut; Different Profile for surface and Cross-Section}
    \label{fig:EDM-Cut}
\end{figure}

The samples were cut in EDM. To identify Surface and Cross-Section samples, an additional groove was made on all the Cross-Section samples. 
\\
The Surface and the Cross-Section microstructure are taken from the same sample. The Cross-Section has a thin line to differentiate it from the Surface.


\begin{figure}[H]
    \centering
    \begin{subfigure}{0.40\textwidth}
        \includegraphics[width=\textwidth]{\Photos{"Ti6242-1-EDM Cut-1.1.1-Top"}}
        \caption{Ti-6242 EDM Cut; Surface}
        \label{fig:2a}
    \end{subfigure}
    ~
    \begin{subfigure}{0.40\textwidth}
        \includegraphics[width=\textwidth]{\Photos{"Ti6242-1-EDM Cut-1.1.1-CS"}}
        \caption{Ti-6242 EDM Cut; Cross-Section}
        \label{fig:2b}
    \end{subfigure}
    %\caption{}
    \label{fig:Ti-6242 EDM Cut; Cross-Section}
\end{figure}

\subsection{Slow Speed Cutting of Cross-Section Sample}

\begin{figure}[H]
    \centering
        \includegraphics[width=\textwidth]{\Photos{"Slow Speed Cutting"}}
        \caption{Ti-6246 Slow Speed Cutting}
    \label{fig:slow-speed-Cut}
\end{figure}



\section{Initial Microstructure}
Polishing - 600, 800, 1000, 1200, 1500, 2000 grit size. \\
Etchant - Equal parts of Methanol + HF + HCl + HNO$_{3}$ \\
Etching Time - 5 seconds

\subsection{Ti-6246}

\begin{figure}[H]
    \centering
    \begin{subfigure}{0.49\textwidth}
        \includegraphics[width=\textwidth]{\HeatTreatment{"Ti6246-1.1-Top (500x)"}}
        \caption{Ti-6246 Surface; Area Fraction - 33.87\%}
        \label{fig:a-As-Received-micro}
    \end{subfigure}
    ~
    \begin{subfigure}{0.49\textwidth}
        \includegraphics[width=\textwidth]{\HeatTreatment{"Ti6246-1.1.1-CS (500x)"}}
        \caption{Ti-6246 Cross-Section; Area Fraction - 56.79\%}
        \label{fig:b-As-Received-micro}
    \end{subfigure}
  
    \caption{Microstructure of Ti-6246 Alloy; Magnification - 500x}
    \label{fig:As-Received-micro}
\end{figure}

\subsection{Ti-6242}

\begin{figure}[H]
    \centering
    \begin{subfigure}{0.49\textwidth}
        \includegraphics[width=\textwidth]{\HeatTreatment{Ti6242-1.2-Top-5(500x).jpg}}
        \caption{Ti-6242 Surface}
        \label{fig:Ti-6242 Surface}
    \end{subfigure}
    ~
    \begin{subfigure}{0.49\textwidth}
        \includegraphics[width=\textwidth]{\HeatTreatment{"Ti6242-1.1-CS-5(500x)"}}
        \caption{Ti-6242 Cross-Section}
        \label{fig:Ti-6242 Cross-Section}
    \end{subfigure}
  
    \caption{Microstructure of Ti-6242 Alloy; Magnification - 500x}
    \label{fig:As-Received}
\end{figure}

\iffalse
\begin{figure}[H]
    \centering
    \begin{subfigure}{0.49\textwidth}
        \includegraphics[width=\textwidth]{\HeatTreatment{"Ti6246-1.1-Top (1000x)"}}
        \caption{Ti-6246 Surface}
        \label{fig:2a}
    \end{subfigure}
    ~
    \begin{subfigure}{0.49\textwidth}
        \includegraphics[width=\textwidth]{\HeatTreatment{"Ti6246-1.1.1-CS (1000x)"}}
        \caption{Ti-6246 Cross-Section}
        \label{fig:2a}
    \end{subfigure}
  
    \caption{SEM Microstructure of Ti-6246 Alloy; Magnification - 1000x}
    \label{fig:As-Received-SEM}
\end{figure}

\fi

\section{Area Fraction Analysis}
Area fraction is given by:
\begin{equation}
V_{v} = \dfrac{\sum A_{A}}{A_{T}}
\end{equation} 

Error is given by:
\begin{equation}
E_{A}^{2} = \dfrac{1}{N}\left[ 1+ \left( \dfrac{\sigma}{\overline A} \right)^{2} \right]
\end{equation} 

\subsection{ImageJ}

\subsubsection{Ti-6246}
\begin{figure}[H]
    \centering
    \begin{subfigure}{0.49\textwidth}
        \includegraphics[width=\textwidth]{\HeatTreatment{"Ti6246-1.1.1-CS (500x) - Threshold"}}
        \caption{Ti-6246 Threshold}
        \label{fig:Ti-6246 Threshold}
    \end{subfigure}
    ~
    \begin{subfigure}{0.49\textwidth}
        \includegraphics[width=\textwidth]{\HeatTreatment{"Ti6246-1.1.1-CS (500x) - Outline"}}
        \caption{Ti-6246 Outline}
        \label{fig:Ti-6246 Outline}
    \end{subfigure}
  
    \caption{Microstructure of Ti-6246 Alloy; Magnification - 500x}
    \label{fig:As-Received-SEM}
\end{figure}

% Area fraction of $\alpha$-phase was measured to be 34.7\% and 
Area fraction of equiaxed $\alpha$-phase was measured to be 23.6\%.
\\
Mean Area - 22.58 $\mu m^{2}$, Standard Deviation - 34.81 $\mu m^{2}$, E$_{A}$ - 0.285 (95\% Accuracy - ie., 2E$_{A}$). 

\subsubsection{Ti-6242}
\begin{figure}[H]
    \centering
    \begin{subfigure}{0.49\textwidth}
        \includegraphics[width=\textwidth]{\HeatTreatment{"Ti6242-1.1-CS-5(500x) - Threshold"}}
        \caption{Ti-6242 Threshold}
        \label{fig:Ti-6242 Threshold}
    \end{subfigure}
    ~
    \begin{subfigure}{0.49\textwidth}
        \includegraphics[width=\textwidth]{\HeatTreatment{"Ti6242-1.1-CS-5(500x) - Outline"}}
        \caption{Ti-6242 Outline}
        \label{fig:Ti-6242 Outline}
    \end{subfigure}
  
    \caption{Microstructure of Ti-6242 Alloy; Magnification - 500x}
    \label{fig:As-Received}
\end{figure}


%The area fraction of $\alpha$-phase is 67.4\%. \\
The area fraction of equiaxed $\alpha$-phase is 27.9\% \\
Mean Area - 397.17 $\mu m^{2}$, Standard Deviation - 275.56 $\mu m^{2}$, E$_{A}$ - 0.703 (95\% Accuracy - ie., 2E$_{A}$).

\subsection{Matlab}

\subsubsection{Ti-6246}

\begin{figure}[H]
    \centering
    \begin{subfigure}{0.49\textwidth}
        \includegraphics[width=\textwidth]{\HeatTreatment{"Ti6246-1.1.1-CS (500x)-MatLab"}}
        \caption{Ti-6246 MatLab}
        \label{fig:Ti-6246 Threshold}
    \end{subfigure}
    ~
    \begin{subfigure}{0.49\textwidth}
        \includegraphics[width=\textwidth]{\HeatTreatment{"Ti6246-1.1.1-CS (500x)-MatLab-Threshold"}}
        \caption{Ti-6246 MatLab Threshold}
        \label{fig:Ti-6246 Outline}
    \end{subfigure}
  
    \caption{Microstructure of Ti-6246 Alloy; Magnification - 500x}
    \label{fig:As-Received-SEM}
\end{figure}

% Area fraction of $\alpha$-phase was measured to be 34.7\% and 
Area fraction of equiaxed $\alpha$-phase was measured to be 19.8\%.
\\
Mean Area - 22.58 $\mu m^{2}$, Standard Deviation - 34.81 $\mu m^{2}$, E$_{A}$ - 0.285 (95\% Accuracy - ie., 2E$_{A}$). 


\section{Tensile Test}

\begin{table}[H]
\centering
\resizebox{\textwidth}{!}{%
\begin{tabular}{ccccccc}
\hline
\textbf{Sample ID} & \textbf{Diameter (mm)} & \textbf{Gauge Length (mm)} & \textbf{Strain Rate (s$ ^{-1} $)} & \textbf{Cross-head Speed (mm/s)} & \textbf{Yield Strength (MPa)} \\ \hline
\hypertarget{Ti6242-1.2-TS}{Ti6242-1.2-TS} & - & - & - & - & & \\ \hline
Ti6242-1.3-TS & - & - & - & - & & \\ \hline
Ti6242-1.4-TS & 4.043 & 20 & 0.0333 & 0.666 & 956.53 @0.2\% \\ \hline
Ti6242-1.5-TS & 4.037 & 20 & 0.0333 & 0.666 & 961.63 @0.2\% \\ \hline
\end{tabular}%
}
\caption{Tensile Test of Ti-6242 Alloy}
\label{table:Tensile-Test-Ti-6242}
\end{table}

Example table row hyperlinking - \hyperlink{Ti6242-1.2-TS}{Test}. %https://tex.stackexchange.com/a/356939

\begin{table}[H]
\centering
\resizebox{\textwidth}{!}{%
\begin{tabular}{c|c|c|}
\cline{2-3}
 & \textbf{My Work} & \textbf{Arunima's Work} \\ \hline
\multicolumn{1}{|c|}{\textbf{Test Type}} & Tensile Test & Compression Test \\ \hline
\multicolumn{1}{|c|}{\textbf{Dimensions}} & \multicolumn{1}{l|}{4 mm (Diameter), 20 mm (Gauge Length)} & \multicolumn{1}{l|}{6 mm x 6 mm x 11 mm (Aspect Ratio - 1.83)} \\ \hline
\multicolumn{1}{|c|}{\textbf{Yield Strength (MPa)}} & 960 (@0.2\%) & 902 (@0.2\%, Thesis), 1045 (@5\%, Presentation) \\ \hline
\multicolumn{1}{|c|}{\textbf{Strain Rate (s$ ^{-1} $)}} & 3.3 x 10$^{-2}$ & 10$^{-3}$ \\ \hline
\end{tabular}%
}
\caption{Yield Strength Comparison with Arunima's Work}
\label{table:Yield Strength Comparison with Arunima's Work}
\end{table}

\begin{figure}[H]
    \centering
    \begin{subfigure}{0.40\textwidth}
        \includegraphics[width=\textwidth]{\TensileTest{Ti6242-1.4-TS-Graph}}
        \caption{Ti6242-1.4 As Received Tensile Test}
        \label{fig:2a}
    \end{subfigure}
    ~
    \begin{subfigure}{0.40\textwidth}
        \includegraphics[width=\textwidth]{\TensileTest{Ti6242-1.5-TS-Graph}}
        \caption{Ti6242-1.5 As Received Tensile Test}
        \label{fig:2b}
    \end{subfigure}
    \\
    \begin{subfigure}{0.40\textwidth}
        \includegraphics[width=\textwidth]{\TensileTest{Tensile Test [Arunima Banerjee]}}
        \caption{Ti6242 As received Tensile Test by Arunima} 
        \label{fig:2b}
    \end{subfigure} 
     
    \caption{Tensile Test of as received Ti-6242 @0.0333 s$ ^{-1} $}
    \label{fig:Tensile-Test-6242}
\end{figure}

%%%%%%%%%%%%%%%%%%%%%%%%%%%% Fatigue Test %%%%%%%%%%%%%%%%%%%%%%%%%%%%%%
\section{Fatigue Test}
\label{sec:Fatigue Test}

\subsection{Ti-6242}
\label{subsec:Fatigue Test -  Ti-6242}

\subsubsection{S-N Curve}
\label{subsec:Fatigue Test -  Ti-6242 - S-N Curve}


\begin{figure}[H]
    \centering
        \includegraphics[width=\textwidth]{\FatigueTest{S-N_Curve.eps}}
        \caption{S-N Curve (Ti-6242 Fatigue Test)}
    \label{fig:fatigue-test-sn-curve}
\end{figure}

%%%%%%%%%%%%%%%%%%%%%%%%%%%% Heat Treatment %%%%%%%%%%%%%%%%%%%%%%%%%%%%%%
\section{Heat Treatment}
\label{sec:Heat Treatment}

\subsection{Ti-6242}

\subsubsection{Beta Transus Temperature}


\begin{figure}[H]
    \centering
    \begin{subfigure}{0.49\textwidth}
        \includegraphics[width=\textwidth]{\HeatTreatment{Ti6242-Heat-Graph.eps}}
        \caption{Heating Cycle}
        \label{fig:Ti-6242 Threshold}
    \end{subfigure}
    ~
    \begin{subfigure}{0.49\textwidth}
        \includegraphics[width=\textwidth]{\HeatTreatment{Ti6242-Cool-Graph.eps}}
        \caption{Cooling Cycle}
        \label{fig:Ti-6242 HT700}
    \end{subfigure}    
   
    \caption{DSC data for Ti-6242 @10 kpm}
    \label{fig:DSC data for Ti-6242 @10 kpm}
\end{figure}

\iffalse
\subsubsection{Microstructure}

\begin{figure}[H]
    \centering
    \begin{subfigure}{0.49\textwidth}
        \includegraphics[width=\textwidth]{\HeatTreatment{"Ti6242-1.1-CS-5(500x)-MatLab"}}
        \caption{Ti-6242 Threshold}
        \label{fig:Ti-6242 Threshold}
    \end{subfigure}
    ~
    \begin{subfigure}{0.49\textwidth}
        \includegraphics[width=\textwidth]{\HeatTreatment{Ti6242-1.1.5-Top-HT701-5-(200x).jpg}}
        \caption{701\degC (700\degC)}
        \label{fig:Ti-6242 HT700}
    \end{subfigure}    
    \\
    \begin{subfigure}{0.49\textwidth}
        \includegraphics[width=\textwidth]{\HeatTreatment{Ti6242-1.1.4-Top-HT752-5-(200x).jpg}}
        \caption{752\degC (750\degC)}
        \label{fig:Ti-6242 HT752}
    \end{subfigure}   
    ~
    \begin{subfigure}{0.49\textwidth}
        \includegraphics[width=\textwidth]{\HeatTreatment{Ti6242-1.1.3-Top-(HT815)-5-(200x).jpg}}
        \caption{815\degC (800\degC)}
        \label{fig:Ti-6242 HT815}
    \end{subfigure}
    \\
    \begin{subfigure}{0.49\textwidth}
        \includegraphics[width=\textwidth]{\HeatTreatment{Ti6242-1.1.2-Top-HT863-5-(200x).jpg}}
        \caption{863\degC (850\degC)}
        \label{fig:Ti-6242 HT863}
    \end{subfigure}
    ~
    \begin{subfigure}{0.49\textwidth}
        \includegraphics[width=\textwidth]{\HeatTreatment{Ti6242-1.1-Top-HT900(2)-17-(500x)}}
        \caption{920\degC (900\degC)}
        \label{fig:Ti-6242 HT920}
    \end{subfigure}        
   
    \caption{Microstructure of Ti-6242 Alloy at different heat treatments; Magnification - 200x}
    \label{fig:As-Received}
\end{figure}


\textbf{Note:} The area fraction of Ti-6242 measured at 100x magnification gave the lowest error and 500x magnification gave the highest error. So all measurements were made at 100x.

\fi

\begin{figure}[H]
    \centering
        \includegraphics[width=\textwidth]{\HeatTreatment{Voulme_Fraction.eps}}
        \caption{Volume Fraction at different temperatures for Ti-6242}
    \label{fig:Volume Fraction at different temperatures for Ti-6242}
\end{figure}


\begin{table}[H]
\centering
\caption{Volume Fraction for Ti-6242}
\label{my-label}
\begin{tabular}{@{}cccccc@{}}
\toprule
\multirow{2}{*}{\textbf{Date}} & \multirow{2}{*}{\textbf{Sample ID}} & \multirow{2}{*}{\textbf{Temperature (\degC)}} & \multicolumn{2}{c}{\textbf{Areal Fraction (\%)}}              & \multirow{2}{*}{\textbf{Pg. No.}} \\ \cmidrule(lr){4-5}
                               &                                     &                                               & \textbf{Surface (E$_{A}$)} & \textbf{Cross-Section (E$_{A}$)} &                                   \\ \midrule
                               & Ti6242-1.1.1                        & As Received                                   & 31.41 (5.6)                & 25.5 (4.8)                       &                                   \\
03 Dec 2017                    & Ti6242-1.1.1                        & 920 (900)                                     & 42.17 (8.72)               &                                  & 29                                \\
04 Dec 2017                    & Ti6242-1.1.2                        & 863 (850)                                     & 38.62 (10.21)              &                                  & 29                                \\
05 Dec 2017                    & Ti6242-1.1.3                        & 815 (800)                                     & 34.38 (8.77)               &                                  & 30                                \\
09 Dec 2017                    & Ti6242-1.1.4                        & 768 (750)                                     & 31.78 (7.24)               &                                  & 31                                \\
10 Dec 2017                    & Ti6242-1.1.5                        & 701 (700)                                     & 32.05 (5.18)               &                                  & 32                                \\ \bottomrule
\end{tabular}
\end{table}

\subsubsection{Grain Size}

\begin{figure}[H]
    \centering
    \begin{subfigure}{0.30\textwidth}
        \includegraphics[width=\textwidth]{\HeatTreatment{Ti6242-1.1.1-Top-10-(500x).jpg}}
        \caption{As Received}
        \label{fig:Grain Size Ti-6242 As Received}
    \end{subfigure}     
	~
    \begin{subfigure}{0.30\textwidth}
        \includegraphics[width=\textwidth]{\HeatTreatment{Ti6242-1.2.1-Top-HT768(2.5)-11-(500x).jpg}}
        \caption{2 Hours}
        \label{fig:Grain Size Ti-6242 2 Hours}
    \end{subfigure}   
    ~
    \begin{subfigure}{0.30\textwidth}
        \includegraphics[width=\textwidth]{\HeatTreatment{Ti6242-1.2.3-Top-HT768(16)-3-(500x).jpg}}
        \caption{16 Hours}
        \label{fig:Grain Size Ti-6242 16 Hours}
    \end{subfigure}
   
    \caption{Grain Size @ 768 \degC (Water Quenched)}
    \label{fig:As-Received}
\end{figure}