\chapter{Experimental Details}
\section{Titanium 6246 Alloy} 

\begin{figure}[H]
    \centering
    \begin{subfigure}{0.40\textwidth}
        \includegraphics[width=\textwidth]{\Photos "Ti-6246 [uncut]"}
        \caption{Ti-6246 As Received; Surface}
        \label{fig:2a}
    \end{subfigure}
    ~
    \begin{subfigure}{0.40\textwidth}
        \includegraphics[width=\textwidth]{\Photos "Ti-6246 [uncut-cs]"}
        \caption{Ti-6246 As Received; Cross-Section}
        \label{fig:2b}
    \end{subfigure}
    %\caption{}
    \label{fig:As-Received}
\end{figure}

\section{Titanium 6242 Alloy} 

\begin{figure}[H]
    \centering
        \includegraphics[width=0.50\textwidth]{\Microstructure "Ti6242-1-Flat"}
        \caption{Ti-6242 As Received; Surface}
    \label{fig:EDM-Cut}
\end{figure}


\subsection{EDM Cutting}
The samples were cut in EDM. To identify Surface and Cross-Section samples, an additional L-shaped groove was made on all the Cross-Section samples. 

\begin{figure}[H]
    \centering
        \includegraphics[width=\textwidth]{\Microstructure "Ti-6246 [cut-EDM]"}
        \caption{Ti-6246 EDM Cut; Different Profile for surface and Cross-Section}
    \label{fig:EDM-Cut}
\end{figure}

The samples were cut in EDM. To identify Surface and Cross-Section samples, an additional groove was made on all the Cross-Section samples. 
\\
The Surface and the Cross-Section microstructure are taken from the same sample. The Cross-Section has a thin line to differentiate it from the Surface.


\begin{figure}[H]
    \centering
    \begin{subfigure}{0.40\textwidth}
        \includegraphics[width=\textwidth]{\Microstructure "Ti6242-1-EDM Cut-1.1.1-Top"}
        \caption{Ti-6242 EDM Cut; Surface}
        \label{fig:2a}
    \end{subfigure}
    ~
    \begin{subfigure}{0.40\textwidth}
        \includegraphics[width=\textwidth]{\Microstructure "Ti6242-1-EDM Cut-1.1.1-CS"}
        \caption{Ti-6242 EDM Cut; Cross-Section}
        \label{fig:2b}
    \end{subfigure}
    %\caption{}
    \label{fig:Ti-6242 EDM Cut; Cross-Section}
\end{figure}

\subsection{Slow Speed Cutting of Cross-Section Sample}

\begin{figure}[H]
    \centering
        \includegraphics[width=\textwidth]{\Microstructure "Slow Speed Cutting"}
        \caption{Ti-6246 Slow Speed Cutting}
    \label{fig:slow-speed-Cut}
\end{figure}



\section{Initial Microstructure}
Polishing - 600, 800, 1000, 1200, 1500, 2000 grit size. \\
Etchant - Equal parts of Methanol + HF + HCl + HNO$_{3}$ \\
Etching Time - 5 seconds

\subsection{Ti-6246}

\begin{figure}[H]
    \centering
    \begin{subfigure}{0.49\textwidth}
        \includegraphics[width=\textwidth]{\Microstructure "Ti6246-1.1-Top (500x)"}
        \caption{Ti-6246 Surface; Area Fraction - 33.87\%}
        \label{fig:a-As-Received-micro}
    \end{subfigure}
    ~
    \begin{subfigure}{0.49\textwidth}
        \includegraphics[width=\textwidth]{\Microstructure "Ti6246-1.1.1-CS (500x)"}
        \caption{Ti-6246 Cross-Section; Area Fraction - 56.79\%}
        \label{fig:b-As-Received-micro}
    \end{subfigure}
  
    \caption{Microstructure of Ti-6246 Alloy; Magnification - 500x}
    \label{fig:As-Received-micro}
\end{figure}

\subsection{Ti-6242}

\begin{figure}[H]
    \centering
    \begin{subfigure}{0.49\textwidth}
        \includegraphics[width=\textwidth]{\Microstructure "Ti6242-1.2-Top-5(500x)"}
        \caption{Ti-6242 Surface}
        \label{fig:Ti-6242 Surface}
    \end{subfigure}
    ~
    \begin{subfigure}{0.49\textwidth}
        \includegraphics[width=\textwidth]{\Microstructure "Ti6242-1.1-CS-5(500x)"}
        \caption{Ti-6242 Cross-Section}
        \label{fig:Ti-6242 Cross-Section}
    \end{subfigure}
  
    \caption{Microstructure of Ti-6242 Alloy; Magnification - 500x}
    \label{fig:As-Received}
\end{figure}

\iffalse
\begin{figure}[H]
    \centering
    \begin{subfigure}{0.49\textwidth}
        \includegraphics[width=\textwidth]{\Microstructure "Ti6246-1.1-Top (1000x)"}
        \caption{Ti-6246 Surface}
        \label{fig:2a}
    \end{subfigure}
    ~
    \begin{subfigure}{0.49\textwidth}
        \includegraphics[width=\textwidth]{\Microstructure "Ti6246-1.1.1-CS (1000x)"}
        \caption{Ti-6246 Cross-Section}
        \label{fig:2a}
    \end{subfigure}
  
    \caption{SEM Microstructure of Ti-6246 Alloy; Magnification - 1000x}
    \label{fig:As-Received-SEM}
\end{figure}

\fi

\section{Area Fraction Analysis}
Area fraction is given by:
\begin{equation}
V_{v} = \dfrac{\sum A_{A}}{A_{T}}
\end{equation} 

Error is given by:
\begin{equation}
E_{A}^{2} = \dfrac{1}{N}\left[ 1+ \left( \dfrac{\sigma}{\overline A} \right)^{2} \right]
\end{equation} 

\subsection{ImageJ}

\subsubsection{Ti-6246}
\begin{figure}[H]
    \centering
    \begin{subfigure}{0.49\textwidth}
        \includegraphics[width=\textwidth]{\Microstructure "Ti6246-1.1.1-CS (500x) - Threshold"}
        \caption{Ti-6246 Threshold}
        \label{fig:Ti-6246 Threshold}
    \end{subfigure}
    ~
    \begin{subfigure}{0.49\textwidth}
        \includegraphics[width=\textwidth]{\Microstructure "Ti6246-1.1.1-CS (500x) - Outline"}
        \caption{Ti-6246 Outline}
        \label{fig:Ti-6246 Outline}
    \end{subfigure}
  
    \caption{Microstructure of Ti-6246 Alloy; Magnification - 500x}
    \label{fig:As-Received-SEM}
\end{figure}

% Area fraction of $\alpha$-phase was measured to be 34.7\% and 
Area fraction of equiaxed $\alpha$-phase was measured to be 23.6\%.
\\
Mean Area - 22.58 $\mu m^{2}$, Standard Deviation - 34.81 $\mu m^{2}$, E$_{A}$ - 0.285 (95\% Accuracy - ie., 2E$_{A}$). 
\\
\subsubsection{Ti-6242}
\begin{figure}[H]
    \centering
    \begin{subfigure}{0.49\textwidth}
        \includegraphics[width=\textwidth]{\Microstructure "Ti6242-1.1-CS-5(500x) - Threshold"}
        \caption{Ti-6242 Threshold}
        \label{fig:Ti-6242 Threshold}
    \end{subfigure}
    ~
    \begin{subfigure}{0.49\textwidth}
        \includegraphics[width=\textwidth]{\Microstructure "Ti6242-1.1-CS-5(500x) - Outline"}
        \caption{Ti-6242 Outline}
        \label{fig:Ti-6242 Outline}
    \end{subfigure}
  
    \caption{Microstructure of Ti-6242 Alloy; Magnification - 500x}
    \label{fig:As-Received}
\end{figure}


%The area fraction of $\alpha$-phase is 67.4\%. \\
The area fraction of equiaxed $\alpha$-phase is 27.9\% \\
Mean Area - 397.17 $\mu m^{2}$, Standard Deviation - 275.56 $\mu m^{2}$, E$_{A}$ - 0.703 (95\% Accuracy - ie., 2E$_{A}$).

\subsection{Matlab}

\subsubsection{Ti-6246}

\begin{figure}[H]
    \centering
    \begin{subfigure}{0.49\textwidth}
        \includegraphics[width=\textwidth]{\Microstructure "Ti6246-1.1.1-CS (500x)-MatLab"}
        \caption{Ti-6246 MatLab}
        \label{fig:Ti-6246 Threshold}
    \end{subfigure}
    ~
    \begin{subfigure}{0.49\textwidth}
        \includegraphics[width=\textwidth]{\Microstructure "Ti6246-1.1.1-CS (500x)-MatLab-Threshold"}
        \caption{Ti-6246 MatLab Threshold}
        \label{fig:Ti-6246 Outline}
    \end{subfigure}
  
    \caption{Microstructure of Ti-6246 Alloy; Magnification - 500x}
    \label{fig:As-Received-SEM}
\end{figure}

% Area fraction of $\alpha$-phase was measured to be 34.7\% and 
Area fraction of equiaxed $\alpha$-phase was measured to be 19.8\%.
\\
Mean Area - 22.58 $\mu m^{2}$, Standard Deviation - 34.81 $\mu m^{2}$, E$_{A}$ - 0.285 (95\% Accuracy - ie., 2E$_{A}$). 
\\

\subsubsection{Ti-6242}
\begin{figure}[H]
    \centering
    \begin{subfigure}{0.49\textwidth}
        \includegraphics[width=\textwidth]{\Microstructure "Ti6242-1.1-CS-5(500x)-MatLab"}
        \caption{Ti-6242 Threshold}
        \label{fig:Ti-6242 Threshold}
    \end{subfigure}
    ~
    \begin{subfigure}{0.49\textwidth}
        \includegraphics[width=\textwidth]{\Microstructure "Ti6242-1.1-CS-5(500x)-MatLab-Threshold"}
        \caption{Ti-6242 Threshold MatLab}
        \label{fig:Ti-6242 Threshold MatLab}
    \end{subfigure}
  
    \caption{Microstructure of Ti-6242 Alloy; Magnification - 500x}
    \label{fig:As-Received}
\end{figure}


The area fraction of equiaxed $\alpha$-phase is 33.34\% \\
Mean Area - 397.17 $\mu m^{2}$, Standard Deviation - 275.56 $\mu m^{2}$, E$_{A}$ - 0.703 (95\% Accuracy - ie., 2E$_{A}$).

\iffalse

\section{Grain Size}
\subsection{Crystallite Size By XRD}

\begin{figure}[H]
    \centering
    \begin{subfigure}{0.49\textwidth}
        \includegraphics[width=\textwidth]{"WH-Ti-6246"}
        \caption{Ti-6246 Williamson-Hall Plot}
        \label{fig:Ti-6246 Williamson-Hall Plot}
    \end{subfigure}
    ~
    \begin{subfigure}{0.49\textwidth}
        \includegraphics[width=\textwidth]{"WH-Ti-6242"}
        \caption{Ti-6242 Williamson-Hall Plot}
        \label{fig:Ti-6242 Williamson-Hall Plot}
    \end{subfigure}
  
    \caption{Williamson-Hall for crystallite size measurement}
    \label{fig:As-Received}
\end{figure}

\begin{equation}
Intercept, X = \frac{K \lambda}{L} \\
where \\
K - Shape factor \\
\lambda - Wavelength of X-Ray Source \\
L - Crystallite Size
\end{equation}

\subsubsection{Crystallite size of Ti-6246}
\begin{equation}
L = \frac{0.9*1.54 A^{0}}{0.00342} = 4.052 * 10^{-8}
\end{equation}

\subsubsection{Crystallite size of Ti-6242}
\begin{equation}
L = \frac{0.9*1.54 A^{0}}{0.00668} = 2.075 * 10^{-8}
\end{equation}

\fi

\section{Tensile Test}

\begin{table}[H]
\centering
\resizebox{\textwidth}{!}{%
\begin{tabular}{ccccccc}
\hline
\multicolumn{1}{c}{\multirow{2}{*}{\textbf{Sample ID}}} & \multicolumn{1}{c}{\multirow{2}{*}{\textbf{Diameter (mm)}}} & \multicolumn{1}{c}{\multirow{2}{*}{\textbf{Gauge Length (mm)}}} & \multicolumn{1}{c}{\multirow{2}{*}{\textbf{Strain Rate (s$ ^{-1} $)}}} & \multicolumn{1}{c}{\multirow{2}{*}{\textbf{Cross-head Speed (mm/s)}}} & \multicolumn{2}{c}{\textbf{Yield Strength (MPa)}} \\ \cline{6-7} 
\multicolumn{1}{c}{} & \multicolumn{1}{c}{} & \multicolumn{1}{c}{} & \multicolumn{1}{c}{} & \multicolumn{1}{c}{} & \multicolumn{1}{c}{\textbf{Engineering}} & \multicolumn{1}{c}{\textbf{True}} \\ \hline
Ti6242-1.2-TS & - & - & - & - & & \\ \hline
Ti6242-1.3-TS & - & - & - & - & & \\ \hline
Ti6242-1.4-TS & 4.043 & 20 & 0.0333 & 0.666 & 956.53 @0.2\% & 1007.13 @0.2\% \\ \hline
Ti6242-1.5-TS & 4.037 & 20 & 0.0333 & 0.666 & 961.63 @0.2\% & 986.75 @0.2\% \\ \hline
\end{tabular}%
}
\caption{Tensile Test of Ti-6242 Alloy}
\label{table:Tensile-Test-Ti-6242}
\end{table}


\begin{table}[H]
\centering
\resizebox{\textwidth}{!}{%
\begin{tabular}{c|c|c|}
\cline{2-3}
 & \textbf{My Work} & \textbf{Arunima's Work} \\ \hline
\multicolumn{1}{|c|}{\textbf{Test Type}} & Tensile Test & Compression Test \\ \hline
\multicolumn{1}{|c|}{\textbf{Dimensions}} & \multicolumn{1}{l|}{4 mm (Diameter), 20 mm (Gauge Length)} & \multicolumn{1}{l|}{6 mm x 6 mm x 11 mm (Aspect Ratio - 1.83)} \\ \hline
\multicolumn{1}{|c|}{\textbf{Yield Strength (MPa)}} & 997 (True, @0.2\%) & 902 (Thesis, @0.2\%), 1045 (Presentation) (True, @5\%) \\ \hline
\multicolumn{1}{|c|}{\textbf{Strain Rate (s$ ^{-1} $)}} & 3.3 x 10$^{-2}$ & 10$^{-3}$ \\ \hline
\end{tabular}%
}
\caption{Yield Strength Comparison with Arunima's Work}
\label{table:Yield Strength Comparison with Arunima's Work}
\end{table}

\begin{figure}[H]
    \centering
    \begin{subfigure}{0.40\textwidth}
        \includegraphics[width=\textwidth]{"Ti6242-1.4-TS-Graph"}
        \caption{Ti6242-1.4 As Received Tensile Test}
        \label{fig:2a}
    \end{subfigure}
    ~
    \begin{subfigure}{0.40\textwidth}
        \includegraphics[width=\textwidth]{"Ti6242-1.5-TS-Graph"}
        \caption{Ti6242-1.5 As Received Tensile Test}
        \label{fig:2b}
    \end{subfigure}
    \\
    \begin{subfigure}{0.40\textwidth}
        \includegraphics[width=\textwidth]{"Tensile Test [Arunima Banerjee]"}
        \caption{Ti6242-1.5 As Received Tensile Test}
        \label{fig:2b}
    \end{subfigure} 
     
    \caption{Tensile Test of as received Ti-6242 @0.0333 s$ ^{-1} $}
    \label{fig:Tensile-Test-6242}
\end{figure}

\section{Fatigue Test}

\begin{figure}[H]
    \centering
        \includegraphics[width=0.50\textwidth]{"Fatigue [Arunima Banerjee]"}
        \caption{Ti-6242 Fatigue Test by Arunima}
    \label{fig:fatigue-test-arunima}
\end{figure}