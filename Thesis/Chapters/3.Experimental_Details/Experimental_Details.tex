\chapter{Experimental Details}
\section{Titanium 6246 Alloy} 

\begin{figure}[H]
    \centering
    \begin{subfigure}{0.40\textwidth}
        \includegraphics[width=\textwidth]{"Ti-6246 [uncut]"}
        \caption{Ti-6246 As Received; Surface}
        \label{fig:2a}
    \end{subfigure}
    ~
    \begin{subfigure}{0.40\textwidth}
        \includegraphics[width=\textwidth]{"Ti-6246 [uncut-cs]"}
        \caption{Ti-6246 As Received; Cross-Section}
        \label{fig:2b}
    \end{subfigure}
    %\caption{}
    \label{fig:As-Received}
\end{figure}

\section{Titanium 6242 Alloy} 

\begin{figure}[H]
    \centering
        \includegraphics[width=0.50\textwidth]{"Ti6242-1-Flat"}
        \caption{Ti-6242 As Received; Surface}
    \label{fig:EDM-Cut}
\end{figure}

\subsection{EDM Cutting}
The samples were cut in EDM. To identify Surface and Cross-Section samples, an additional L-shaped groove was made on all the Cross-Section samples. 

\begin{figure}[H]
    \centering
        \includegraphics[width=\textwidth]{"Ti-6246 [cut-EDM]"}
        \caption{Ti-6246 EDM Cut; Different Profile for surface and Cross-Section}
    \label{fig:EDM-Cut}
\end{figure}

The samples were cut in EDM. To identify Surface and Cross-Section samples, an additional groove was made on all the Cross-Section samples. 
\\
The Surface and the Cross-Section microstructure are taken from the same sample. The Cross-Section has a thin line to differentiate it from the Surface.

\begin{figure}[H]
    \centering
    \begin{subfigure}{0.40\textwidth}
        \includegraphics[width=\textwidth]{"Ti6242-1-EDM Cut-1.1.1-Top"}
        \caption{Ti-6242 EDM Cut; Surface}
        \label{fig:2a}
    \end{subfigure}
    ~
    \begin{subfigure}{0.40\textwidth}
        \includegraphics[width=\textwidth]{"Ti6242-1-EDM Cut-1.1.1-CS"}
        \caption{Ti-6242 EDM Cut; Cross-Section}
        \label{fig:2b}
    \end{subfigure}
    %\caption{}
    \label{fig:Ti-6242 EDM Cut; Cross-Section}
\end{figure}


