\documentclass[10pt]{beamer}

\usetheme[progressbar=frametitle]{metropolis}
\usepackage{appendixnumberbeamer}

\usepackage{booktabs}
\usepackage[scale=2]{ccicons}

\usepackage{pgfplots}
\usepgfplotslibrary{dateplot}

\usepackage{xspace}
\newcommand{\themename}{\textbf{\textsc{metropolis}}\xspace}

\usepackage{amsmath}
\usepackage{graphicx}

\title{Quantitative Metallography}
%\subtitle{Sharan Chandran}
\date{\today}
\author{Sharan Chandran}
\institute{Indian Institute of Science}
% \titlegraphic{\hfill\includegraphics[height=1.5cm]{logo.pdf}}

\begin{document}

%% Adds images directory for each chapter
\graphicspath{{../Images/}} 

\maketitle

\begin{frame}{Table of contents}
  \setbeamertemplate{section in toc}[sections numbered]
  \tableofcontents[hideallsubsections]
\end{frame}

\section{Volume Fraction}

\begin{frame}[fragile]{Metropolis}


\end{frame}

\metroset{block=fill}

\begin{frame}{Notations}


  \begin{columns}[T,onlytextwidth]
    \column{0.3\textwidth}
    P Points \\
    L Lines  \\
    P$ _{P} $ Number of points \\
    $\sigma$ Standard Deviation
 
    \column{0.3\textwidth}
   A Flat Areas \\
   S Curved Surfaces \\
   S$ _{V} $ Surface / Volume \\
   $\bar{x}$ or $\mu$ Mean Value
  
    \column{0.3\textwidth}
    V Volume \\
    N Number \\
    V$ _{V} $ Volume of phase
\end{columns}     

\end{frame}



{%
\setbeamertemplate{frame footer}{Slide 14 - https://www.brainshark.com/malvern/vu?pi=577133879\& text=M021507\& r3f1=}
\begin{frame}[fragile]{Theoretical Density}

\begin{equation*}
Theoretical Density = \dfrac{\text{MW * No.of Molecules per unit volume}}{\text{Volume of unit cell * Avogadro number}}
\end{equation*}   
    
\end{frame}
}
%%%%%%%%%%%%%%%%%%%%%%%%%%%%%%%%%%%%%%%%%%%%%%%%%%%%%%%%%%%%%%%%%%%%%%%%%%%%%%%%%%%%
{%
\setbeamertemplate{frame footer}{Slide 14 - https://www.brainshark.com/malvern/vu?pi=577133879\& text=M021507\& r3f1=}
\begin{frame}[fragile]{Volume Fraction}


  \begin{columns}[T,onlytextwidth]
 
    \column{0.49\textwidth}
\begin{block}{Areal Fraction}
\begin{equation*}
V_{v} = \dfrac{\sum A_{A}}{A_{T}}
\end{equation*} 
\end{block}
 
 \column{0.49\textwidth}

\begin{block}{Error}
\begin{equation*}
E_{A}^{2} = \dfrac{1}{N}\left[ 1+ \left( \dfrac{\sigma}{\bar{A}} \right)^{2} \right]
\end{equation*} 
\end{block}

\end{columns}

  \begin{columns}[T,onlytextwidth]
    \column{0.49\textwidth}
\begin{block}{Lineal Fraction}
\begin{equation*}
L_{v} = \dfrac{\sum A_{A}}{A_{T}}
\end{equation*} 
\end{block}
 
 \column{0.49\textwidth}

\begin{block}{Error}
\small
\begin{equation*}
E_{L}^{2} = \dfrac{1}{N}(1-V_{V})^{2}\left[ \left( \dfrac{\sigma^{\alpha}_{L}}{\bar{L_{\alpha}}} \right)^{2} + \left( \dfrac{\sigma^{\beta}_{L}}{\bar{L_{\beta}}} \right)^{2} \right]
\end{equation*}

\end{block}

\end{columns}

  \begin{columns}[T,onlytextwidth]
    \column{0.49\textwidth}
\begin{block}{Point Fraction}
\begin{equation*}
P_{v} = \dfrac{\sum A_{A}}{A_{T}}
\end{equation*} 
\end{block}
 
 \column{0.49\textwidth}

\begin{block}{Error}
\begin{equation*}
E_{P}^{2} = \dfrac{1}{P}
\end{equation*} 
\end{block}

\end{columns}


 
    
\end{frame}
}
%%%%%%%%%%%%%%%%%%%%%%%%%%%%%%%%%%%%%%%%%%%%%%%%%%%%%%%%%%%%%%%%%%%%%%%%%%%%%%%%%%%%
\section{Grain Size}

{%
\setbeamertemplate{frame footer}{}
\begin{frame}[fragile]{Mean Linear Intercept}

Best Method for grain size analysis for uniform grains

\begin{figure}[H]
\includegraphics[width=\textwidth,height=\textheight,keepaspectratio]{"Graphite"}
\end{figure}
    
\end{frame}
}

{%
\setbeamertemplate{frame footer}{}
\begin{frame}[fragile]{Jeffries Planimetric Method}

ASTM Committee E-4 on Metallography was founded in 1916. Their first standard, ASTM E 2 was proposed in 1917 [1], it included a detailed description of how to measure grain size using the planimetric method written by Zay Jeffries [2,3], based upon earlier work done by his graduate school advisor, Albert Sauveur [4]. Sauveur was the first to measure grain size and he defined it in terms of the number of grains per mm2 at 1X; but, he did not describe how he made the measurement. The ASTM equation relating the number of grains per in2 at 100X and the ASTM grain size number was introduced in 1951 by an undocumented Timken member of E-4 when E 91-51T was introduced [5]. The equation is:

n = 2G-1                                            (1)

where n is the number of grains per in2 at 100X and G is the ASTM grain size number.

The Jeffries planimetric grain size method utilized a test circle with a diameter of 79.8 mm which was superimposed over the microstructure. The magnification was chosen to give at least 100 grains to be counted (this number was later reduced to ~50 when a round robin test program showed that counting errors were higher when ~100 grains were counted per grid application). The wording, as written, could be interpreted two ways and could be improved. It would be better to say that ~50 grains should be inside and intersecting the test circle. The rater must count all the grains that are completely inside the test circle, ninside, and all the grains that are intercepted by the circle, nintercepted. It is assumed that, on average, half of the intercepted grains are inside the test circle and half are outside. To get an accurate count, the operator must mark off the grains as they are counted using a felt tip pen, etc. This, however, makes the method slow and less popular. But, this is not a problem if image analysis is used. The calculation is:

NA = f(ninside + 0.5nintercepted)                          (2)

where NA is the number of grains per mm2 at 1X and f is the Jeffries multiplier:

f = M2/A                                                     (3)

and M is the magnification and a is the area (5000 mm2 is the standard size). If the test area is different than 5000 mm2 (from a circle 79.8 mm diameter), then the alternate area used is divided into the magnification squared.

The average grain area, A, is the reciprocal of NA. The ASTM grain size number is calculated by:

G = 3.321928 LogNA  –  2.954            (4)

G is rounded off to the nearest tenth value. In practice, more than one field must be evaluated to obtain a good estimate of G.

\begin{figure}[H]
\includegraphics[width=\textwidth,height=\textheight,keepaspectratio]{"Graphite"}
\end{figure}
    
\end{frame}
}


{%
\setbeamertemplate{frame footer}{H. Abrams, “Grain Size Measurements by the Intercept Method,”Metallography, Vol. 4, February 1971, pp. 59-78}
\begin{frame}[fragile]{Heyn Intercept Method}
url: https://vacaero.com/information-resources/metallography-with-george-vander-voort/1273-grain-size-measurement-the-heyn-intercept-method.html
In the 1974 revision of E 112 by Halle Abrams, he introduced the three-concentric circle test grid and a more formal methodology for performing intercept grain size measurements. His idea [2, 3] was that the total circumference of the three circles was 500 mm and he suggested adjusting the magnification so that, on average, about 100 grain boundary intersections, P, or grain interceptions, N, would be obtained. Then, five random applications of the three-circle grid would yield ~500 N or P hits which would give ~10\% relative accuracy. This strategy was verified in a subsequent interlaboratory round-robin [4]. However, counting errors seemed to be greater when each placement yielded ~100 counts compared to a magnification that yielded ~50 counts. So, the standard was modified to recommend using a magnification that yielded ~50 counts per placement. The goal was still to get ~500 total counts. Ten randomly chosen fields would give a more representative estimate of the grain size than five fields.

To do the intercept method, one counts either grain boundary intersections, P, or grains intercepted, N, by the circles. For a single phase structure, it is easier to do P counts. For a two-phase structure, one must do N counts. For a single phase grain structure P = N and either count can be made. The P or N count is divided by the true line length, LT, which is the line length divided by the magnification, L/M. This yields PL or NL, the number of intersections per unit length or the number of interceptions per unit length. The reciprocal of PL or NL is the mean lineal intercept length, L3, which may be designated as l. The mean lineal intercept is related to G by the following empirical equation:
							G = (-6.6457 log L3)  – 3.298      
    
\end{frame}
}


{%
\setbeamertemplate{frame footer}{Slide 14 - https://www.brainshark.com/malvern/vu?pi=577133879\& text=M021507\& r3f1=}
\begin{frame}[fragile]{Not all grains are created equal}

\begin{equation*}
Theoretical Density = \dfrac{\text{MW * No.of Molecules per unit volume}}{\text{Volume of unit cell * Avogadro number}}
\end{equation*}   
    
\end{frame}
}



%%%%%%%%%%%%%%%%%%%%%%%%%%%%%%%%%%%%%%%%%%%%%%%%%%%%%%%%%%%%%%%%%%%%%%%%%%%%%%%%%%%%
{\section{MatLab}

%%%%%%%%%%%%%%%%%%%%%%%%%%%%%%%%%%%%%%%%%%%%%%%%%%%%%%%%%%%%%%%%%%%%%%%%%%%%%%%%%%%%
{\setbeamercolor{palette primary}{fg=black, bg=yellow}
\begin{frame}[standout]
  Thank You
\end{frame}
}
\iffalse
%%%%%%%%%%%%%%%%%%%%%%%%%%%%%%%%%%%%%%%%%%%%%%%%%%%%%%%%%%%%%%%%%%%%%%%%%%%%%%%%%%%%
\begin{frame}[fragile]{Backup slides}
\begin{figure}[H]
    \centering
    \begin{subfigure}[H]{0.40\textwidth}
        \includegraphics[width=\textwidth]{images/sem-cs-2}
        
    \end{subfigure}
    ~
    \begin{subfigure}[H]{0.40\textwidth}
        \includegraphics[width=\textwidth]{images/sem-cs-4}
    \end{subfigure}       
    \\
    \begin{subfigure}[H]{0.35\textwidth}
        \includegraphics[width=\textwidth]{images/sem-3}
    \end{subfigure}      
     
\end{figure}
\end{frame}
\fi

%%%%%%%%%%%%%%%%%%%%%%%%%%%%%%%%%%%%%%%%%%%%%%%%%%%%%%%%%%%%%%%%%%%%%%%%%%%%%%%%%%%%
\begin{frame}[allowframebreaks]{References}

  %\bibliography{references}
  \bibliographystyle{abbrv}

\end{frame}

\end{document}
