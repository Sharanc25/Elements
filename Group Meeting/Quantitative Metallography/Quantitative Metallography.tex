\documentclass[10pt]{beamer}

\usetheme[progressbar=frametitle]{metropolis}
\usepackage{appendixnumberbeamer}

\usepackage{booktabs}
\usepackage[scale=2]{ccicons}

\usepackage{pgfplots}
\usepgfplotslibrary{dateplot}

\usepackage{xspace}
\newcommand{\themename}{\textbf{\textsc{metropolis}}\xspace}

\usepackage{amsmath}
\usepackage{graphicx}

\title{Quantitative Metallography}
%\subtitle{Sharan Chandran}
\date{\today}
\author{Sharan Chandran}
\institute{Indian Institute of Science}
% \titlegraphic{\hfill\includegraphics[height=1.5cm]{logo.pdf}}

\begin{document}

\maketitle

\begin{frame}{Table of contents}
  \setbeamertemplate{section in toc}[sections numbered]
  \tableofcontents[hideallsubsections]
\end{frame}

\section{Volume Fraction}

\begin{frame}[fragile]{Metropolis}


\end{frame}

\metroset{block=fill}

\begin{frame}{Notations}


  \begin{columns}[T,onlytextwidth]
    \column{0.3\textwidth}
    P Points \\
    L Lines  \\
    P$ _{P} $ Number of points \\
    $\sigma$ Standard Deviation
 
    \column{0.3\textwidth}
   A Flat Areas \\
   S Curved Surfaces \\
   S$ _{V} $ Surface / Volume \\
   $\bar{x}$ or $\mu$ Mean Value
  
    \column{0.3\textwidth}
    V Volume \\
    N Number \\
    V$ _{V} $ Volume of phase
\end{columns}     

\end{frame}



{%
\setbeamertemplate{frame footer}{Slide 14 - https://www.brainshark.com/malvern/vu?pi=577133879\& text=M021507\& r3f1=}
\begin{frame}[fragile]{Theoretical Density}

\begin{equation*}
Theoretical Density = \dfrac{\text{MW * No.of Molecules per unit volume}}{\text{Volume of unit cell * Avogadro number}}
\end{equation*}   
    
\end{frame}
}
%%%%%%%%%%%%%%%%%%%%%%%%%%%%%%%%%%%%%%%%%%%%%%%%%%%%%%%%%%%%%%%%%%%%%%%%%%%%%%%%%%%%
{%
\setbeamertemplate{frame footer}{Slide 14 - https://www.brainshark.com/malvern/vu?pi=577133879\& text=M021507\& r3f1=}
\begin{frame}[fragile]{Volume Fraction}


  \begin{columns}[T,onlytextwidth]
 
    \column{0.49\textwidth}
\begin{block}{Areal Fraction}
\begin{equation*}
V_{v} = \dfrac{\sum A_{A}}{A_{T}}
\end{equation*} 
\end{block}
 
 \column{0.49\textwidth}

\begin{block}{Error}
\begin{equation*}
E_{A}^{2} = \dfrac{1}{N}\left[ 1+ \left( \dfrac{\sigma}{\bar{A}} \right)^{2} \right]
\end{equation*} 
\end{block}

\end{columns}

  \begin{columns}[T,onlytextwidth]
    \column{0.49\textwidth}
\begin{block}{Lineal Fraction}
\begin{equation*}
L_{v} = \dfrac{\sum A_{A}}{A_{T}}
\end{equation*} 
\end{block}
 
 \column{0.49\textwidth}

\begin{block}{Error}
\small
\begin{equation*}
E_{L}^{2} = \dfrac{1}{N}(1-V_{V})^{2}\left[ \left( \dfrac{\sigma^{\alpha}_{L}}{\bar{L_{\alpha}}} \right)^{2} + \left( \dfrac{\sigma^{\beta}_{L}}{\bar{L_{\beta}}} \right)^{2} \right]
\end{equation*}

\end{block}

\end{columns}

  \begin{columns}[T,onlytextwidth]
    \column{0.49\textwidth}
\begin{block}{Point Fraction}
\begin{equation*}
P_{v} = \dfrac{\sum A_{A}}{A_{T}}
\end{equation*} 
\end{block}
 
 \column{0.49\textwidth}

\begin{block}{Error}
\begin{equation*}
E_{P}^{2} = \dfrac{1}{P}
\end{equation*} 
\end{block}

\end{columns}


 
    
\end{frame}
}
%%%%%%%%%%%%%%%%%%%%%%%%%%%%%%%%%%%%%%%%%%%%%%%%%%%%%%%%%%%%%%%%%%%%%%%%%%%%%%%%%%%%
\section{Grain Size}

{%
\setbeamertemplate{frame footer}{}
\begin{frame}[fragile]{Mean Linear Intercept}

Best Method for grain size analysis for uniform grains
    
\end{frame}
}



{%
\setbeamertemplate{frame footer}{Slide 14 - https://www.brainshark.com/malvern/vu?pi=577133879\& text=M021507\& r3f1=}
\begin{frame}[fragile]{Not all grains are created equal}

\begin{equation*}
Theoretical Density = \dfrac{\text{MW * No.of Molecules per unit volume}}{\text{Volume of unit cell * Avogadro number}}
\end{equation*}   
    
\end{frame}
}



%%%%%%%%%%%%%%%%%%%%%%%%%%%%%%%%%%%%%%%%%%%%%%%%%%%%%%%%%%%%%%%%%%%%%%%%%%%%%%%%%%%%
{\section{MatLab}

%%%%%%%%%%%%%%%%%%%%%%%%%%%%%%%%%%%%%%%%%%%%%%%%%%%%%%%%%%%%%%%%%%%%%%%%%%%%%%%%%%%%
{\setbeamercolor{palette primary}{fg=black, bg=yellow}
\begin{frame}[standout]
  Thank You
\end{frame}
}

\appendix
%%%%%%%%%%%%%%%%%%%%%%%%%%%%%%%%%%%%%%%%%%%%%%%%%%%%%%%%%%%%%%%%%%%%%%%%%%%%%%%%%%%%
\begin{frame}[fragile]{Backup slides}
\begin{figure}[H]
    \centering
    \begin{subfigure}[H]{0.40\textwidth}
        \includegraphics[width=\textwidth]{images/sem-cs-2}
        
    \end{subfigure}
    ~
    \begin{subfigure}[H]{0.40\textwidth}
        \includegraphics[width=\textwidth]{images/sem-cs-4}
    \end{subfigure}       
    \\
    \begin{subfigure}[H]{0.35\textwidth}
        \includegraphics[width=\textwidth]{images/sem-3}
    \end{subfigure}      
     
\end{figure}
\end{frame}

%%%%%%%%%%%%%%%%%%%%%%%%%%%%%%%%%%%%%%%%%%%%%%%%%%%%%%%%%%%%%%%%%%%%%%%%%%%%%%%%%%%%
\begin{frame}[allowframebreaks]{References}

  %\bibliography{references}
  \bibliographystyle{abbrv}

\end{frame}

\end{document}
