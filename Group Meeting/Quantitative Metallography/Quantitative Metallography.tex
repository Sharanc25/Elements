\documentclass[10pt]{beamer}

\usetheme[progressbar=frametitle]{metropolis}
\usepackage{appendixnumberbeamer}

\usepackage{booktabs}
\usepackage[scale=2]{ccicons}

\usepackage{pgfplots}
\usepgfplotslibrary{dateplot}

\usepackage{xspace}
\newcommand{\themename}{\textbf{\textsc{metropolis}}\xspace}

\usepackage{amsmath}
\usepackage{graphicx}

\title{Quantitative Metallography}
%\subtitle{Sharan Chandran}
\date{\today}
\author{Sharan Chandran}
\institute{Indian Institute of Science}
% \titlegraphic{\hfill\includegraphics[height=1.5cm]{logo.pdf}}

\begin{document}

%% Adds images directory for each chapter
\graphicspath{{../Images/}} 

\maketitle

\begin{frame}{Table of contents}
  \setbeamertemplate{section in toc}[sections numbered]
  \tableofcontents[hideallsubsections]
\end{frame}

\section{Volume Fraction}

\begin{frame}[fragile]{Metropolis}


\end{frame}

\metroset{block=fill}

\begin{frame}{Notations}


  \begin{columns}[T,onlytextwidth]
    \column{0.3\textwidth}
    P Points \\
    L Lines  \\
    P$ _{P} $ Number of points \\
    $\sigma$ Standard Deviation
 
    \column{0.3\textwidth}
   A Flat Areas \\
   S Curved Surfaces \\
   S$ _{V} $ Surface / Volume \\
   $\bar{x}$ or $\mu$ Mean Value
  
    \column{0.3\textwidth}
    V Volume \\
    N Number \\
    V$ _{V} $ Volume of phase
\end{columns}     

\end{frame}



{%
\setbeamertemplate{frame footer}{Slide 14 - https://www.brainshark.com/malvern/vu?pi=577133879\& text=M021507\& r3f1=}
\begin{frame}[fragile]{Theoretical Density}

\begin{equation*}
Theoretical Density = \dfrac{\text{MW * No.of Molecules per unit volume}}{\text{Volume of unit cell * Avogadro number}}
\end{equation*}   
    
\end{frame}
}
%%%%%%%%%%%%%%%%%%%%%%%%%%%%%%%%%%%%%%%%%%%%%%%%%%%%%%%%%%%%%%%%%%%%%%%%%%%%%%%%%%%%
{%
\setbeamertemplate{frame footer}{Slide 14 - https://www.brainshark.com/malvern/vu?pi=577133879\& text=M021507\& r3f1=}
\begin{frame}[fragile]{Volume Fraction}


  \begin{columns}[T,onlytextwidth]
 
    \column{0.49\textwidth}
\begin{block}{Areal Fraction}
\begin{equation*}
V_{v} = \dfrac{\sum A_{A}}{A_{T}}
\end{equation*} 
\end{block}
 
 \column{0.49\textwidth}

\begin{block}{Error}
\begin{equation*}
E_{A}^{2} = \dfrac{1}{N}\left[ 1+ \left( \dfrac{\sigma}{\bar{A}} \right)^{2} \right]
\end{equation*} 
\end{block}

\end{columns}

  \begin{columns}[T,onlytextwidth]
    \column{0.49\textwidth}
\begin{block}{Lineal Fraction}
\begin{equation*}
L_{v} = \dfrac{\sum A_{A}}{A_{T}}
\end{equation*} 
\end{block}
 
 \column{0.49\textwidth}

\begin{block}{Error}
\small
\begin{equation*}
E_{L}^{2} = \dfrac{1}{N}(1-V_{V})^{2}\left[ \left( \dfrac{\sigma^{\alpha}_{L}}{\bar{L_{\alpha}}} \right)^{2} + \left( \dfrac{\sigma^{\beta}_{L}}{\bar{L_{\beta}}} \right)^{2} \right]
\end{equation*}

\end{block}

\end{columns}

  \begin{columns}[T,onlytextwidth]
    \column{0.49\textwidth}
\begin{block}{Point Fraction}
\begin{equation*}
P_{v} = \dfrac{\sum A_{A}}{A_{T}}
\end{equation*} 
\end{block}
 
 \column{0.49\textwidth}

\begin{block}{Error}
\begin{equation*}
E_{P}^{2} = \dfrac{1}{P}
\end{equation*} 
\end{block}

\end{columns}


 
    
\end{frame}
}
%%%%%%%%%%%%%%%%%%%%%%%%%%%%%%%%%%%%%%%%%%%%%%%%%%%%%%%%%%%%%%%%%%%%%%%%%%%%%%%%%%%%
\section{Grain Size}

{%
\setbeamertemplate{frame footer}{}
\begin{frame}[fragile]{Mean Linear Intercept}

Best Method for grain size analysis for uniform grains

\begin{figure}[H]
\includegraphics[width=\textwidth,height=\textheight,keepaspectratio]{"Graphite"}
\end{figure}
    
\end{frame}
}

{%
\setbeamertemplate{frame footer}{}
\begin{frame}[fragile]{Mean Linear Intercept}

Best Method for grain size analysis for uniform grains

\begin{figure}[H]
\includegraphics[width=\textwidth,height=\textheight,keepaspectratio]{"Graphite"}
\end{figure}
    
\end{frame}
}


{%
\setbeamertemplate{frame footer}{H. Abrams, “Grain Size Measurements by the Intercept Method,”Metallography, Vol. 4, February 1971, pp. 59-78}
\begin{frame}[fragile]{Heyn Intercept Method}
url: https://vacaero.com/information-resources/metallography-with-george-vander-voort/1273-grain-size-measurement-the-heyn-intercept-method.html
In the 1974 revision of E 112 by Halle Abrams, he introduced the three-concentric circle test grid and a more formal methodology for performing intercept grain size measurements. His idea [2, 3] was that the total circumference of the three circles was 500 mm and he suggested adjusting the magnification so that, on average, about 100 grain boundary intersections, P, or grain interceptions, N, would be obtained. Then, five random applications of the three-circle grid would yield ~500 N or P hits which would give ~10\% relative accuracy. This strategy was verified in a subsequent interlaboratory round-robin [4]. However, counting errors seemed to be greater when each placement yielded ~100 counts compared to a magnification that yielded ~50 counts. So, the standard was modified to recommend using a magnification that yielded ~50 counts per placement. The goal was still to get ~500 total counts. Ten randomly chosen fields would give a more representative estimate of the grain size than five fields.

To do the intercept method, one counts either grain boundary intersections, P, or grains intercepted, N, by the circles. For a single phase structure, it is easier to do P counts. For a two-phase structure, one must do N counts. For a single phase grain structure P = N and either count can be made. The P or N count is divided by the true line length, LT, which is the line length divided by the magnification, L/M. This yields PL or NL, the number of intersections per unit length or the number of interceptions per unit length. The reciprocal of PL or NL is the mean lineal intercept length, L3, which may be designated as l. The mean lineal intercept is related to G by the following empirical equation:
							G = (-6.6457 log L3)  – 3.298      
    
\end{frame}
}


{%
\setbeamertemplate{frame footer}{Slide 14 - https://www.brainshark.com/malvern/vu?pi=577133879\& text=M021507\& r3f1=}
\begin{frame}[fragile]{Not all grains are created equal}

\begin{equation*}
Theoretical Density = \dfrac{\text{MW * No.of Molecules per unit volume}}{\text{Volume of unit cell * Avogadro number}}
\end{equation*}   
    
\end{frame}
}



%%%%%%%%%%%%%%%%%%%%%%%%%%%%%%%%%%%%%%%%%%%%%%%%%%%%%%%%%%%%%%%%%%%%%%%%%%%%%%%%%%%%
{\section{MatLab}

%%%%%%%%%%%%%%%%%%%%%%%%%%%%%%%%%%%%%%%%%%%%%%%%%%%%%%%%%%%%%%%%%%%%%%%%%%%%%%%%%%%%
{\setbeamercolor{palette primary}{fg=black, bg=yellow}
\begin{frame}[standout]
  Thank You
\end{frame}
}
\iffalse
%%%%%%%%%%%%%%%%%%%%%%%%%%%%%%%%%%%%%%%%%%%%%%%%%%%%%%%%%%%%%%%%%%%%%%%%%%%%%%%%%%%%
\begin{frame}[fragile]{Backup slides}
\begin{figure}[H]
    \centering
    \begin{subfigure}[H]{0.40\textwidth}
        \includegraphics[width=\textwidth]{images/sem-cs-2}
        
    \end{subfigure}
    ~
    \begin{subfigure}[H]{0.40\textwidth}
        \includegraphics[width=\textwidth]{images/sem-cs-4}
    \end{subfigure}       
    \\
    \begin{subfigure}[H]{0.35\textwidth}
        \includegraphics[width=\textwidth]{images/sem-3}
    \end{subfigure}      
     
\end{figure}
\end{frame}
\fi

%%%%%%%%%%%%%%%%%%%%%%%%%%%%%%%%%%%%%%%%%%%%%%%%%%%%%%%%%%%%%%%%%%%%%%%%%%%%%%%%%%%%
\begin{frame}[allowframebreaks]{References}

  %\bibliography{references}
  \bibliographystyle{abbrv}

\end{frame}

\end{document}
