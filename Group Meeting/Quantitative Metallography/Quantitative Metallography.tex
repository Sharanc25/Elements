\documentclass[10pt]{beamer}

\usetheme[progressbar=frametitle]{metropolis}
\usepackage{appendixnumberbeamer}

\usepackage{booktabs}
\usepackage[scale=2]{ccicons}

\usepackage{pgfplots}
\usepgfplotslibrary{dateplot}

\usepackage{xspace}
\newcommand{\themename}{\textbf{\textsc{metropolis}}\xspace}

\usepackage{amsmath}
\usepackage{graphicx}
\usepackage{subcaption}

\title{Quantitative Metallography}
%\subtitle{Sharan Chandran}
\date{\today}
\author{Sharan Chandran}
\institute{Indian Institute of Science}
% \titlegraphic{\hfill\includegraphics[height=1.5cm]{logo.pdf}}

\begin{document}

%% Adds images directory for each chapter
\graphicspath{{../Images/}} 

\maketitle

\metroset{block=fill}

\begin{frame}{Table of contents}
  \setbeamertemplate{section in toc}[sections numbered]
  \tableofcontents[hideallsubsections]
\end{frame}

\section{Volume Fraction}

\begin{frame}{Notations}


  \begin{columns}[T,onlytextwidth]
    \column{0.3\textwidth}
    P Points \\
    L Lines  \\
    P$ _{P} $ Number of points \\
    $\sigma$ Standard Deviation
 
    \column{0.3\textwidth}
   A Flat Areas \\
   S Curved Surfaces \\
   S$ _{V} $ Surface/Volume \\
   $\bar{x}$ or $\mu$ Mean Value
  
    \column{0.3\textwidth}
    V Volume \\
    N Number \\
    V$ _{V} $ Volume of phase
\end{columns}     

\end{frame}
%%%%%%%%%%%%%%%%%%%%%%%%%%%%%%%%%%%%%%%%%%%%%%%%%%%%%%%%%%%%%%%%%%%%%%%%%%%%%%%%%%%%
{%
\setbeamertemplate{frame footer}{Underwood, E.E., 1969. Stereology, or the quantitative evaluation of microstructures. Journal of microscopy, 89(2), pp.161-180.}
\begin{frame}[fragile]{2D vs 3D Microstructure}

\begin{figure}[H]
\includegraphics[width=\textwidth,height=\textheight,keepaspectratio]{"Graphite"}

\end{figure}
    
\end{frame}
}
%%%%%%%%%%%%%%%%%%%%%%%%%%%%%%%%%%%%%%%%%%%%%%%%%%%%%%%%%%%%%%%%%%%%%%%%%%%%%%%%%%%%
{%
\setbeamertemplate{frame footer}{Underwood, E.E., 1972. The stereology of projected images. Journal of microscopy, 95(1), pp.25-44}
\begin{frame}[fragile]{Projected Image}

\begin{figure}[H]
\includegraphics[width=0.90\textwidth,height=\textheight,keepaspectratio]{"Projected Image"}

\end{figure}
    
\end{frame}
}

{%
\setbeamertemplate{frame footer}{Underwood, E.E., 1972. The stereology of projected images. Journal of microscopy, 95(1), pp.25-44}
\begin{frame}[fragile]{Projected Image}

\begin{figure}[H]
\includegraphics[width=0.90\textwidth,height=\textheight,keepaspectratio]{"Projection of Structural Features"}

\end{figure}
    
\end{frame}
}
%%%%%%%%%%%%%%%%%%%%%%%%%%%%%%%%%%%%%%%%%%%%%%%%%%%%%%%%%%%%%%%%%%%%%%%%%%%%%%%%%%%%
{%
\setbeamertemplate{frame footer}{}
\begin{frame}[fragile]{Volume Fraction}

\begin{equation*}
\begin{gathered}
V_{V} = A_{A} = L_{L} = P_{P} \\
W_{W} = \dfrac{V_{V} * \rho_{\alpha}}{\rho_{M}}
\end{gathered}
\end{equation*}


  \begin{columns}[T,onlytextwidth]
 
    \column{0.49\textwidth}
\begin{block}{Areal Fraction}
\begin{equation*}
A_{A} = \dfrac{\sum A_{\alpha}}{A_{T}}
\end{equation*} 
\end{block}
 
 \column{0.49\textwidth}

\begin{block}{Error}
\begin{equation*}
E_{A}^{2} = \dfrac{1}{N}\left[ 1+ \left( \dfrac{\sigma}{\bar{A}} \right)^{2} \right]
\end{equation*} 
\end{block}

\end{columns}

  \begin{columns}[T,onlytextwidth]
    \column{0.49\textwidth}
\begin{block}{Lineal Fraction}
\begin{equation*}
L_{L} = \dfrac{\sum L_{\alpha}}{L_{T}}
\end{equation*} 
\end{block}
 
 \column{0.49\textwidth}

\begin{block}{Error}
\small
\begin{equation*}
E_{L}^{2} = \dfrac{1}{N}(1-V_{V})^{2}\left[ \left( \dfrac{\sigma^{\alpha}_{L}}{\bar{L_{\alpha}}} \right)^{2} + \left( \dfrac{\sigma^{\beta}_{L}}{\bar{L_{\beta}}} \right)^{2} \right]
\end{equation*}

\end{block}

\end{columns}

  \begin{columns}[T,onlytextwidth]
    \column{0.49\textwidth}
\begin{block}{Point Fraction (ASTM E 562)}
\begin{equation*}
P_{P} = \dfrac{\sum P_{\alpha}}{A_{T}}
\end{equation*} 
\end{block}
 
 \column{0.49\textwidth}

\begin{block}{Error}
\begin{equation*}
E_{P}^{2} = \dfrac{1}{P}
\end{equation*} 
\end{block}

\end{columns}


 
    
\end{frame}
}
%%%%%%%%%%%%%%%%%%%%%%%%%%%%%%%%%%%%%%%%%%%%%%%%%%%%%%%%%%%%%%%%%%%%%%%%%%%%%%%%%%%%
\section{Grain Size}

{%
\setbeamertemplate{frame footer}{Z. Jeffries, A.H. Kline and E.B. Zimmer, Trans. AIME, Vol. 54, 1916, pp. 594-607}
\begin{frame}[fragile]{Jeffries Planimetric Method}

\begin{figure}[H]
\includegraphics[width=\textwidth,height=\textheight,keepaspectratio]{"Jefferies"}
\end{figure}

where n is the number of grains per in2 at 100X and G is the ASTM grain size number.

The Jeffries planimetric grain size method utilized a test circle with a diameter of 79.8 mm which was superimposed over the microstructure. The magnification was chosen to give at least 100 grains to be counted (this number was later reduced to ~50 when a round robin test program showed that counting errors were higher when ~100 grains were counted per grid application). The wording, as written, could be interpreted two ways and could be improved. It would be better to say that ~50 grains should be inside and intersecting the test circle. The rater must count all the grains that are completely inside the test circle, ninside, and all the grains that are intercepted by the circle, nintercepted. It is assumed that, on average, half of the intercepted grains are inside the test circle and half are outside. To get an accurate count, the operator must mark off the grains as they are counted using a felt tip pen, etc. This, however, makes the method slow and less popular. But, this is not a problem if image analysis is used. The calculation is:

NA = f(ninside + 0.5nintercepted)                          (2)

where NA is the number of grains per mm2 at 1X and f is the Jeffries multiplier:

f = M2/A                                                     (3)

and M is the magnification and a is the area (5000 mm2 is the standard size). If the test area is different than 5000 mm2 (from a circle 79.8 mm diameter), then the alternate area used is divided into the magnification squared.

The average grain area, A, is the reciprocal of NA. The ASTM grain size number is calculated by:

G = 3.321928 LogNA  –  2.954            (4)

G is rounded off to the nearest tenth value. In practice, more than one field must be evaluated to obtain a good estimate of G.


    
\end{frame}
}


{%
\setbeamertemplate{frame footer}{H. Abrams, “Grain Size Measurements by the Intercept Method,”Metallography, Vol. 4, February 1971, pp. 59-78}
\begin{frame}[fragile]{Heyn Intercept Method}
url: https://vacaero.com/information-resources/metallography-with-george-vander-voort/1273-grain-size-measurement-the-heyn-intercept-method.html
In the 1974 revision of E 112 by Halle Abrams, he introduced the three-concentric circle test grid and a more formal methodology for performing intercept grain size measurements. His idea [2, 3] was that the total circumference of the three circles was 500 mm and he suggested adjusting the magnification so that, on average, about 100 grain boundary intersections, P, or grain interceptions, N, would be obtained. Then, five random applications of the three-circle grid would yield ~500 N or P hits which would give ~10\% relative accuracy. This strategy was verified in a subsequent interlaboratory round-robin [4]. However, counting errors seemed to be greater when each placement yielded ~100 counts compared to a magnification that yielded ~50 counts. So, the standard was modified to recommend using a magnification that yielded ~50 counts per placement. The goal was still to get ~500 total counts. Ten randomly chosen fields would give a more representative estimate of the grain size than five fields.

To do the intercept method, one counts either grain boundary intersections, P, or grains intercepted, N, by the circles. For a single phase structure, it is easier to do P counts. For a two-phase structure, one must do N counts. For a single phase grain structure P = N and either count can be made. The P or N count is divided by the true line length, LT, which is the line length divided by the magnification, L/M. This yields PL or NL, the number of intersections per unit length or the number of interceptions per unit length. The reciprocal of PL or NL is the mean lineal intercept length, L3, which may be designated as l. The mean lineal intercept is related to G by the following empirical equation:
							G = (-6.6457 log L3)  – 3.298      
							
							
							Measurement of Grain Size Using the Mean Lineal Intercept Method
ASTM Standard E 112-88 describes the lineal intercept method in general and the Heyn procedure
in particular. The following is an abbreviated version of the Heyn procedure.
1. Magnification, Test Line and Fields
C Ideally, the length of a single test line and the magnification should be such that 50 intercepts
can be counted.
C The use of multiple fields to obtain 50 intercepts is discouraged, although permitted, due to the
bias which will decrease the accuracy as the number of fields increases.
2. Count the intercepts...
C An intercept is the segment of the test line which overlays one grain.
C Count 1 for each intercept and ½ for each time an end of the test line falls in a grain.
... or count the intersections
C An intersection is the point where the test line cuts a grain boundary.
C Count 1 for each intersection, 1 for each tangential intersection, ½ when an end of the test line
ends exactly on a grain boundary and 1½ when the intersection occurs at a triple point.
3. Measurement Strategy
C Make counts on 3 to 5 blindly selected and widely separate fields to obtain a reasonable average
for the specimen. Additional counts may be required to obtain the desired statistics.
C Use four or more orientations of the test line to eliminate effects due the moderate departures
from an equiaxed structure.
C Use orthogonal sets of parallel test lines if the structure is distinctly non-equiaxed.
4. Tabulate the Results
C Calculate the average lineal intercept length.
C Calculate the standard deviation.
C Calculate the confidence interval for the desired confidence interval.
5. Report the Results
C When reporting the results include the magnification, length of the test line, total number of
intercepts counted along with the mean intercept length and the confidence interval for the stated
confidence level.
    
\end{frame}
}


{%
\setbeamertemplate{frame footer}{}
\begin{frame}[fragile]{Mean Lineal Intercept}

S$ _{V} $ is the amount of grain surface per unit volume.

\begin{figure}[H]
\includegraphics[width=\textwidth,height=\textheight,keepaspectratio]{"Mean Lineal Intercept"}
\end{figure}
    
\end{frame}
}

{%
\setbeamertemplate{frame footer}{}
\begin{frame}[fragile]{Other Methods}

S$ _{V} $ is the amount of grain surface per unit volume.

\begin{figure}[H]
\includegraphics[width=\textwidth,height=\textheight,keepaspectratio]{"Mean Lineal Intercept"}
\end{figure}
    
\end{frame}
}

{%
\setbeamertemplate{frame footer}{}
\begin{frame}[fragile]{Not all grains are created equal}

    
\end{frame}
}



%%%%%%%%%%%%%%%%%%%%%%%%%%%%%%%%%%%%%%%%%%%%%%%%%%%%%%%%%%%%%%%%%%%%%%%%%%%%%%%%%%%%
%\section{Calculating Volume Fraction Using MatLab}

%%%%%%%%%%%%%%%%%%%%%%%%%%%%%%%%%%%%%%%%%%%%%%%%%%%%%%%%%%%%%%%%%%%%%%%%%%%%%%%%%%%%
{\setbeamercolor{palette primary}{fg=black, bg=yellow}
\begin{frame}[standout]
  Thank You
\end{frame}
}
\iffalse
%%%%%%%%%%%%%%%%%%%%%%%%%%%%%%%%%%%%%%%%%%%%%%%%%%%%%%%%%%%%%%%%%%%%%%%%%%%%%%%%%%%%
\begin{frame}[fragile]{Backup slides}
\begin{figure}[H]
    \centering
    \begin{subfigure}[H]{0.40\textwidth}
        \includegraphics[width=\textwidth]{images/sem-cs-2}
        
    \end{subfigure}
    ~
    \begin{subfigure}[H]{0.40\textwidth}
        \includegraphics[width=\textwidth]{images/sem-cs-4}
    \end{subfigure}       
    \\
    \begin{subfigure}[H]{0.35\textwidth}
        \includegraphics[width=\textwidth]{images/sem-3}
    \end{subfigure}      
     
\end{figure}
\end{frame}
\fi

%%%%%%%%%%%%%%%%%%%%%%%%%%%%%%%%%%%%%%%%%%%%%%%%%%%%%%%%%%%%%%%%%%%%%%%%%%%%%%%%%%%%
\begin{frame}[allowframebreaks]{References}

  %\bibliography{references}
  \bibliographystyle{abbrv}

\end{frame}

\end{document}
