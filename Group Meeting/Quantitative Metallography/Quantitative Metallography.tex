\documentclass[10pt]{beamer}

\usetheme[progressbar=frametitle]{metropolis}
\usepackage{appendixnumberbeamer}

\usepackage{booktabs}
\usepackage[scale=2]{ccicons}

\usepackage{pgfplots}
\usepgfplotslibrary{dateplot}

\usepackage{xspace}
\newcommand{\themename}{\textbf{\textsc{metropolis}}\xspace}

\usepackage{amsmath}
\usepackage{graphicx}
\usepackage{subcaption}
\usepackage{cancel} % For strikethrough

\title{Quantitative Metallography}
\subtitle{Part 1}
\date{\today}
\author{Sharan Chandran}
\institute{Indian Institute of Science}
% \titlegraphic{\hfill\includegraphics[height=1.5cm]{logo.pdf}}

\begin{document}

%% Adds images directory for each chapter
\graphicspath{{../Images/}} 

\maketitle

\metroset{block=fill}

\begin{frame}{Table of contents}
  \setbeamertemplate{section in toc}[sections numbered]
  \tableofcontents[hideallsubsections]
\end{frame}

\section{Volume Fraction}

\begin{frame}{Notations}


  \begin{columns}[T,onlytextwidth]
    \column{0.3\textwidth}
    P Points \\
    L Lines  \\
    P$ _{P} $ Number of points \\
    $\sigma$ Standard Deviation
 
    \column{0.3\textwidth}
   A Flat Areas \\
   S Curved Surfaces \\
   S$ _{V} $ Surface/Volume \\
   $\bar{x}$ or $\mu$ Mean Value
  
    \column{0.3\textwidth}
    V Volume \\
    N Number \\
    V$ _{V} $ Volume of phase
\end{columns}     

\end{frame}
%%%%%%%%%%%%%%%%%%%%%%%%%%%%%%%%%%%%%%%%%%%%%%%%%%%%%%%%%%%%%%%%%%%%%%%%%%%%%%%%%%%%
{%
\setbeamertemplate{frame footer}{Underwood, E.E., 1969. Stereology, or the quantitative evaluation of microstructures. Journal of microscopy, 89(2), pp.161-180.}
\begin{frame}[fragile]{2D vs 3D Microstructure}

\begin{figure}[H]
\includegraphics[width=\textwidth,height=\textheight,keepaspectratio]{"Graphite"}

\end{figure}
    
\end{frame}
}
%%%%%%%%%%%%%%%%%%%%%%%%%%%%%%%%%%%%%%%%%%%%%%%%%%%%%%%%%%%%%%%%%%%%%%%%%%%%%%%%%%%%
{%
\setbeamertemplate{frame footer}{Underwood, E.E., 1972. The stereology of projected images. Journal of microscopy, 95(1), pp.25-44}
\begin{frame}[fragile]{Projected Image}

\begin{figure}[H]
\includegraphics[width=0.90\textwidth,height=\textheight,keepaspectratio]{"Projected Image"}

\end{figure}
    
\end{frame}
}

{%
\setbeamertemplate{frame footer}{Underwood, E.E., 1972. The stereology of projected images. Journal of microscopy, 95(1), pp.25-44}
\begin{frame}[fragile]{Projected Image}

\begin{figure}[H]
\includegraphics[width=0.90\textwidth,height=\textheight,keepaspectratio]{"Projection of Structural Features"}

\end{figure}
    
\end{frame}
}
%%%%%%%%%%%%%%%%%%%%%%%%%%%%%%%%%%%%%%%%%%%%%%%%%%%%%%%%%%%%%%%%%%%%%%%%%%%%%%%%%%%%
{%
\setbeamertemplate{frame footer}{}
\begin{frame}[fragile]{Volume Fraction}

\begin{equation*}
\begin{gathered}
V_{V} = A_{A} = L_{L} = P_{P} \\
W_{W} = \dfrac{V_{V} * \rho_{\alpha}}{\rho_{M}}
\end{gathered}
\end{equation*}


  \begin{columns}[T,onlytextwidth]
 
    \column{0.49\textwidth}
\begin{block}{Areal Fraction}
\begin{equation*}
A_{A} = \dfrac{\sum A_{\alpha}}{A_{T}}
\end{equation*} 
\end{block}
 
 \column{0.49\textwidth}

\begin{block}{Error}
\begin{equation*}
E_{A}^{2} = \dfrac{1}{N}\left[ 1+ \left( \dfrac{\sigma}{\bar{A}} \right)^{2} \right]
\end{equation*} 
\end{block}

\end{columns}

  \begin{columns}[T,onlytextwidth]
    \column{0.49\textwidth}
\begin{block}{Lineal Fraction}
\begin{equation*}
L_{L} = \dfrac{\sum L_{\alpha}}{L_{T}}
\end{equation*} 
\end{block}
 
 \column{0.49\textwidth}

\begin{block}{Error}
\small
\begin{equation*}
E_{L}^{2} = \dfrac{1}{N}(1-V_{V})^{2}\left[ \left( \dfrac{\sigma^{\alpha}_{L}}{\bar{L_{\alpha}}} \right)^{2} + \left( \dfrac{\sigma^{\beta}_{L}}{\bar{L_{\beta}}} \right)^{2} \right]
\end{equation*}

\end{block}

\end{columns}

  \begin{columns}[T,onlytextwidth]
    \column{0.49\textwidth}
\begin{block}{Point Fraction (ASTM E 562)}
\begin{equation*}
P_{P} = \dfrac{\sum P_{\alpha}}{A_{T}}
\end{equation*} 
\end{block}
 
 \column{0.49\textwidth}

\begin{block}{Error}
\begin{equation*}
E_{P}^{2} = \dfrac{1}{P}
\end{equation*} 
\end{block}

\end{columns}


 
    
\end{frame}
}
%%%%%%%%%%%%%%%%%%%%%%%%%%%%%%%%%%%%%%%%%%%%%%%%%%%%%%%%%%%%%%%%%%%%%%%%%%%%%%%%%%%%
{%
\setbeamertemplate{frame footer}{Slide 14 - https://www.brainshark.com/malvern/vu?pi=577133879\& text=M021507\& r3f1=}
\begin{frame}[fragile]{Theoretical Density}

\begin{equation*}
Theoretical Density = \dfrac{\text{MW * No.of Molecules per unit volume}}{\text{Volume of unit cell * Avogadro number}}
\end{equation*}   
    
\end{frame}
}
%%%%%%%%%%%%%%%%%%%%%%%%%%%%%%%%%%%%%%%%%%%%%%%%%%%%%%%%%%%%%%%%%%%%%%%%%%%%%%%%%%%%
\section{Grain Size}

{%
\setbeamertemplate{frame footer}{}
\begin{frame}[fragile]{ASTM Grain Size}

\begin{equation*}
n = 2^{G-1}
\end{equation*}
n - Total number of grains/sq.inch at 100x \\
G - ASTM Grain Size Number    
\end{frame}
}

{%
\setbeamertemplate{frame footer}{Z. Jeffries, A.H. Kline and E.B. Zimmer, Trans. AIME, Vol. 54, 1916, pp. 594-607}
\begin{frame}[fragile]{Jeffries Planimetric Method (ASTM E112)}

\begin{figure}[H]
\includegraphics[width=0.75\textwidth,height=\textheight,keepaspectratio]{"Jefferies"}
\end{figure}
\end{frame}
}

{%
\setbeamertemplate{frame footer}{Z. Jeffries, A.H. Kline and E.B. Zimmer, Trans. AIME, Vol. 54, 1916, pp. 594-607}
\begin{frame}[fragile]{Jeffries Planimetric Method (ASTM E112)}
Circle Diameter = 79.8 mm. \\
Magnification - 50 (grains inside + intersecting the circle) \\
n$ _{inside} $ - Number of grains inside the circle \\
n$_{intercepted}$ - Number of grains intercepted by the circle

\begin{equation*}
N_{A} = f(n_{inside} + 0.5n_{intercepted})  
\end{equation*}
N$_{A}$ - Number of grains per mm$^{2}$ at 1X \\ 
f - Jeffries multiplier

\begin{equation*}
f = M_{2}/A                                                     
\end{equation*}
M - magnification \\
a - area of the circle \\

The average grain area, A, is the reciprocal of N$ _{A} $.
   
\end{frame}
}

{%
\setbeamertemplate{frame footer}{H. Abrams, “Grain Size Measurements by the Intercept Method,”Metallography, Vol. 4, February 1971, pp. 59-78}
\begin{frame}[fragile]{Mean Lineal Intercept}

S$ _{V} $ is the amount of grain surface per unit volume.

\begin{figure}[H]
\includegraphics[width=0.75\textwidth,height=\textheight,keepaspectratio]{"Mean Lineal Intercept"}
\end{figure}
    
\end{frame}
}

{%
\setbeamertemplate{frame footer}{Quantitative metallography and stereology lecture - H. Badeshia}
\begin{frame}[fragile]{Mistakes in Intercept Method}

\begin{figure}[H]
\includegraphics[width=0.75\textwidth,height=\textheight,keepaspectratio]{"Grain Size Error"}
\end{figure}
    
\end{frame}
}
\iffalse
\section{Grain Size Distribution*}
{%
\setbeamertemplate{frame footer}{}
\begin{frame}[fragile]{Not all grains are created equal}

1. Distribution of diameters \\

 - Saltykov's Method \\
 - Scheil's Method \\
 
  
 
2. Distribution of areas \\

3. Distribution of chord lengths \\
     
\end{figure}
    
\end{frame}
}

\fi

%%%%%%%%%%%%%%%%%%%%%%%%%%%%%%%%%%%%%%%%%%%%%%%%%%%%%%%%%%%%%%%%%%%%%%%%%%%%%%%%%%%%
%\section{Calculating Volume Fraction Using MatLab}

%%%%%%%%%%%%%%%%%%%%%%%%%%%%%%%%%%%%%%%%%%%%%%%%%%%%%%%%%%%%%%%%%%%%%%%%%%%%%%%%%%%%
{\setbeamercolor{palette primary}{fg=black, bg=yellow}
\begin{frame}[standout]
  Thank You
\end{frame}
}
\iffalse
%%%%%%%%%%%%%%%%%%%%%%%%%%%%%%%%%%%%%%%%%%%%%%%%%%%%%%%%%%%%%%%%%%%%%%%%%%%%%%%%%%%%
\begin{frame}[fragile]{Backup slides}
\begin{figure}[H]
    \centering
    \begin{subfigure}[H]{0.40\textwidth}
        \includegraphics[width=\textwidth]{images/sem-cs-2}
        
    \end{subfigure}
    ~
    \begin{subfigure}[H]{0.40\textwidth}
        \includegraphics[width=\textwidth]{images/sem-cs-4}
    \end{subfigure}       
    \\
    \begin{subfigure}[H]{0.35\textwidth}
        \includegraphics[width=\textwidth]{images/sem-3}
    \end{subfigure}      
     
\end{figure}
\end{frame}
\fi

%%%%%%%%%%%%%%%%%%%%%%%%%%%%%%%%%%%%%%%%%%%%%%%%%%%%%%%%%%%%%%%%%%%%%%%%%%%%%%%%%%%%
\begin{frame}[allowframebreaks]{References}

  %\bibliography{references}
  \bibliographystyle{abbrv}

\end{frame}

\end{document}
