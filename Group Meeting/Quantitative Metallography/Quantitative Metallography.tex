\documentclass[10pt]{beamer}

\usetheme[progressbar=frametitle]{metropolis}
\usepackage{appendixnumberbeamer}

\usepackage{booktabs}
\usepackage[scale=2]{ccicons}

\usepackage{pgfplots}
\usepgfplotslibrary{dateplot}

\usepackage{xspace}
\newcommand{\themename}{\textbf{\textsc{metropolis}}\xspace}

\usepackage{amsmath}

\title{Quantitative Metallography}
\subtitle{Quantitative Metallography}
\date{\today}
\date{}
\author{Sharan Chandran}
\institute{Indian Institute of Science}
% \titlegraphic{\hfill\includegraphics[height=1.5cm]{logo.pdf}}

\begin{document}

\maketitle

\begin{frame}{Table of contents}
  \setbeamertemplate{section in toc}[sections numbered]
  \tableofcontents[hideallsubsections]
\end{frame}

\section{Introduction}

\begin{frame}[fragile]{Metropolis}
Todo: \\
1. Common edge detection algorithms include Sobel, Canny, Prewitt, Roberts, and fuzzy logic methods. [https://in.mathworks.com/discovery/edge-detection.html] \\
2. https://in.mathworks.com/discovery/image-segmentation.html - https://in.mathworks.com/help/images/examples/correcting-nonuniform-illumination.html \\
3. 
\end{frame}

\metroset{block=fill}

\begin{frame}{Notations}

\begin{block}{Standard Notations}
\hspace{1em}
  \begin{columns}[T,onlytextwidth]
    \column{0.3\textwidth}
    P Points \\
    L Lines  \\
    P$ _{P} $ Number of points
  
 
    \column{0.3\textwidth}
   A Flat Areas \\
   S Curved Surfaces \\
   S$ _{V} $ Surface per unit Volume
  
    \column{0.3\textwidth}
    V Volume \\
    N Number \\
    V$ _{V} $ Volume of phase
\end{columns} 
    
\end{block}

\end{frame}



{%
\setbeamertemplate{frame footer}{Slide 14 - https://www.brainshark.com/malvern/vu?pi=577133879\& text=M021507\& r3f1=}
\begin{frame}[fragile]{Theoretical Density}

\begin{equation*}
Theoretical Density = \dfrac{\text{MW * No.of Molecules per unit volume}}{\text{Volume of unit cell * Avogadro number}}
\end{equation*}   
    
\end{frame}
}
%%%%%%%%%%%%%%%%%%%%%%%%%%%%%%%%%%%%%%%%%%%%%%%%%%%%%%%%%%%%%%%%%%%%%%%%%%%%%%%%%%%%
{%
\setbeamertemplate{frame footer}{Slide 14 - https://www.brainshark.com/malvern/vu?pi=577133879\& text=M021507\& r3f1=}
\begin{frame}[fragile]{Volume Fraction}
Areal Fraction:
  \begin{columns}[T,onlytextwidth]
 
    \column{0.45\textwidth}
\begin{block}{Areal Fraction}
\begin{equation*}
V_{v} = \dfrac{\sum A_{A}}{A_{T}}
\end{equation*} 
\end{block}
 
 \column{0.45\textwidth}

\begin{block}{Error}
\begin{equation*}
E_{A}^{2} = \dfrac{1}{N}\left[ 1+ \left( \dfrac{\sigma}{\bar{A}} \right)^{2} \right]
\end{equation*} 
\end{block}

\end{columns}

Lineal Fraction:
  \begin{columns}[T,onlytextwidth]
    \column{0.45\textwidth}
\begin{block}{Lineal Fraction}
\begin{equation*}
L_{v} = \dfrac{\sum A_{A}}{A_{T}}
\end{equation*} 
\end{block}
 
 \column{0.45\textwidth}

\begin{block}{Error}
\begin{equation*}
E_{A}^{2} = \dfrac{1}{N}\left[ 1+ \left( \dfrac{\sigma}{\bar{A}} \right)^{2} \right]
\end{equation*} 
\end{block}

\end{columns}

Point Fraction:
  \begin{columns}[T,onlytextwidth]
    \column{0.45\textwidth}
\begin{block}{Lineal Fraction}
\begin{equation*}
P_{v} = \dfrac{\sum A_{A}}{A_{T}}
\end{equation*} 
\end{block}
 
 \column{0.45\textwidth}

\begin{block}{Error}
\begin{equation*}
E_{P}^{2} = \dfrac{1}{P}
\end{equation*} 
\end{block}

\end{columns}


 
    
\end{frame}
}
%%%%%%%%%%%%%%%%%%%%%%%%%%%%%%%%%%%%%%%%%%%%%%%%%%%%%%%%%%%%%%%%%%%%%%%%%%%%%%%%%%%%
\section{Titleformats}



%%%%%%%%%%%%%%%%%%%%%%%%%%%%%%%%%%%%%%%%%%%%%%%%%%%%%%%%%%%%%%%%%%%%%%%%%%%%%%%%%%%%
{
    \metroset{titleformat frame=smallcaps}
\begin{frame}{Small caps}

\end{frame}
}

%%%%%%%%%%%%%%%%%%%%%%%%%%%%%%%%%%%%%%%%%%%%%%%%%%%%%%%%%%%%%%%%%%%%%%%%%%%%%%%%%%%%
{
\metroset{titleformat frame=allsmallcaps}
\begin{frame}{All small caps}

\end{frame}
}
%%%%%%%%%%%%%%%%%%%%%%%%%%%%%%%%%%%%%%%%%%%%%%%%%%%%%%%%%%%%%%%%%%%%%%%%%%%%%%%%%%%%
{
\metroset{titleformat frame=allcaps}
\begin{frame}{All caps}

\end{frame}
}

%%%%%%%%%%%%%%%%%%%%%%%%%%%%%%%%%%%%%%%%%%%%%%%%%%%%%%%%%%%%%%%%%%%%%%%%%%%%%%%%%%%%
\begin{frame}{Blocks}
 

      \begin{block}{Default}
        Block content.
      \end{block}

      \begin{alertblock}{Alert}
        Block content.
      \end{alertblock}

      \begin{exampleblock}{Example}
        Block content.
      \end{exampleblock}


\end{frame}

%%%%%%%%%%%%%%%%%%%%%%%%%%%%%%%%%%%%%%%%%%%%%%%%%%%%%%%%%%%%%%%%%%%%%%%%%%%%%%%%%%%%
\begin{frame}{References}
  Some references to showcase [allowframebreaks] \cite{knuth92,ConcreteMath,Simpson,Er01,greenwade93}
\end{frame}
%%%%%%%%%%%%%%%%%%%%%%%%%%%%%%%%%%%%%%%%%%%%%%%%%%%%%%%%%%%%%%%%%%%%%%%%%%%%%%%%%%%%
\section{Conclusion}

\begin{frame}{Summary}

\end{frame}

%%%%%%%%%%%%%%%%%%%%%%%%%%%%%%%%%%%%%%%%%%%%%%%%%%%%%%%%%%%%%%%%%%%%%%%%%%%%%%%%%%%%
{\setbeamercolor{palette primary}{fg=black, bg=yellow}
\begin{frame}[standout]
  Questions?
\end{frame}
}

\appendix
%%%%%%%%%%%%%%%%%%%%%%%%%%%%%%%%%%%%%%%%%%%%%%%%%%%%%%%%%%%%%%%%%%%%%%%%%%%%%%%%%%%%
\begin{frame}[fragile]{Backup slides}
Test
\end{frame}

%%%%%%%%%%%%%%%%%%%%%%%%%%%%%%%%%%%%%%%%%%%%%%%%%%%%%%%%%%%%%%%%%%%%%%%%%%%%%%%%%%%%
\begin{frame}[allowframebreaks]{References}

  \bibliography{references}
  \bibliographystyle{abbrv}

\end{frame}

\end{document}
