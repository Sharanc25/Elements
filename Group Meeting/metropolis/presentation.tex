\documentclass[10pt]{beamer}

\usetheme[progressbar=frametitle]{metropolis}

\usepackage{booktabs}
\usepackage[scale=2]{ccicons}

\usepackage{pgfplots}
\usepgfplotslibrary{dateplot}

\usepackage{xspace}

% Custom Packages
\usepackage{graphicx}
\usepackage{epstopdf}
\usepackage{float}
\usepackage{multirow} %% For Multi-Row functionality in tables
\usepackage[hang]{caption}
\usepackage{subcaption}
\usepackage{wrapfig} % For image float left
\usepackage{mhchem}

\newcommand{\themename}{\textbf{\textsc{metropolis}}\xspace}

\title{STUDY OF ALUMINIUM AS A \\ POTENTIAL NEGATIVE ELECTRODE \\ FOR LITHIUM ION BATTERIES}
\subtitle{Elements Lab \\ Materials Engineering \\ Indian Institute of Science}
\date{\today}
\author{Prachi Pragnya \\ \underline{Under the guidance of} \\ Dr. Vijay A. Sethuraman}
\institute{\underline{Sponsors} \\ Ministry of Human Resources and Development, Government of India
 \\ Indian Institute of Science, Bangalore: Faculty Startup Grant}
\titlegraphic{\hfill\includegraphics[height=1.3cm]{iisclogo}}

\begin{document}

\maketitle
\iffalse
\begin{frame}{Table of contents}
  \setbeamertemplate{section in toc}[sections numbered]
  \tableofcontents[hideallsubsections]
\end{frame}
\fi
%%%%%%%%%%%%%%%%%%%%%%% Introduction %%%%%%%%%%%%%%%%%%%%%%%%%%%
\section{Introduction}

\begin{frame}[fragile]{Objectives}
\begin{enumerate}
\item Study Aluminium-Lithium half cell.
\item Analyze electrochemical as well as mechanical behaviour of bulk and thin film aluminium half cell.
\item Electrochemical set-up for performing electrochemical and stress measurements simultaneously (Beaker Cell).
\item Analyze phase formed during cycling(XRD).
\item Analyze the microstructure and orientation dependent growth of the new phases formed (SEM and EBSD).
\item Estimate percentage straining due to phase transformation. 
\item Develop a Multi-beam Optical Stress Sensor (MOSS) setup for in situ stress meaurements.
\end{enumerate}
\end{frame}

%%%%%%%%%%%%%%%%%%%%%%% Li-ion batteries %%%%%%%%%%%%%%%%%%%%%%%%%%%

{\setbeamertemplate{frame footer}{Tarascon JM, Armand M. Issues and challenges facing rechargeable lithium batteries. Nature. 2001 Nov 15;414(6861):359-67}
\begin{frame}[fragile]{Li-ion batteries}
\begin{figure}[H]
\centering
\includegraphics[width=\textwidth,height=0.75\textheight,keepaspectratio]{images/Li-Energy-Density}
\caption{Comparison on the basis of volumetric and gravimetric energy densities}
\end{figure}
\end{frame}
}

%%%%%%%%%%%%%%%%%%%%%%% Charging and discharging of battery %%%%%%%%%%%%%%%%%%%%%%%%%%%

{\setbeamertemplate{frame footer}{http://www.nexeon.co.uk/about-li-ion-batteries/}
\begin{frame}[fragile]{Charging and discharging of battery}

\begin{figure}[!htbp]
\centering
\includegraphics[width=\textwidth,height=\textheight,keepaspectratio]{images/Battery}
\end{figure}

\begin{center}
\ce{
Al + \textit{x}Li^+ + \textit{x}e^-  <=>[\ce{charging}][\ce{discharging}] \ce{Li_xAl}
}
\end{center}

\end{frame}
}

%%%%%%%%%%%%%%%%%%%%%%% Aluminium as negative electrode %%%%%%%%%%%%%%%%%%%%%%%%%%%

{\setbeamertemplate{frame footer}{Krishnan R, Lu T.M, Koratkar N. Nano Lett., 2011, 11 (2), pp 377–384}
\begin{frame}[fragile]{Aluminium as negative electrode}
\begin{enumerate}
\item High theoretical mass capacity of 2200 mAh/g.
\item Theoretical capacity of 993 mAh/g, corresponding to the formation of LiAl phase, which is considerably higher than that of commercially used graphite negative electrode (372 mAh/g), but much less than capacity for Si (3579 mAh/g).
\item However, Al exhibits approximately 3 times lower volume expansion than Si. 
\end{enumerate}

\begin{figure}[!htbp]
\centering
\includegraphics[width=0.48\textwidth,height=0.48\textheight,keepaspectratio]{images/Al-Expansion}
\caption{Expansion of electrodes}
\end{figure}

\end{frame}
}

%%%%%%%%%%%%%%%%%%%%%%% Li-Al Phase Diagram %%%%%%%%%%%%%%%%%%%%%%%%%%%

\begin{frame}{Li-Al Phase Diagram}

%\begin{wrapfigure}{l}{0.5\textwidth}
\begin{figure}[!htbp]
\centering
\includegraphics[width=0.75\textwidth,height=\textheight,keepaspectratio]{images/Al-Li-phase}
\end{figure}
%\end{wrapfigure}
\small{\textbf{Cycling bulk-Al (poly)}: The first phase is LiAl; the next phase does not occur probably because of lack of 'driving force'

\textbf{Cycling Al thin films}: Here the material runs out and subsequent phases form} 

\end{frame}

%%%%%%%%%%%%%%%%%%%%%%% Stresses created during lithiation and delithiation %%%%%%%%%%%%%%%%%%%%%%%%%%%

\begin{frame}{Stresses created during lithiation and delithiation}

\begin{figure}[!htbp]
\centering
\includegraphics[width=\textwidth,height=\textheight,keepaspectratio]{images/Lithiation}
\end{figure}

\end{frame}

%%%%%%%%%%%%%%%%%%%%%%% Thermodynamics of Batteries %%%%%%%%%%%%%%%%%%%%%%%%%%%

\begin{frame}[fragile]{Thermodynamics of Batteries}

\begin{figure}[H]
    \centering
    \begin{subfigure}[H]{0.30\textwidth}
        \includegraphics[width=\textwidth]{images/gibbs-1}
        \label{fig:fib-1}
    \end{subfigure}
    ~
    \begin{subfigure}[H]{0.30\textwidth}
        \includegraphics[width=\textwidth]{images/gibbs-2}
        \label{fig:fib-2}
    \end{subfigure}
    ~
    \begin{subfigure}[H]{0.30\textwidth}
        \includegraphics[width=\textwidth]{images/gibbs-3}
        \label{fig:fib-2}
    \end{subfigure}  
    \\
    \begin{subfigure}[H]{0.30\textwidth}
        \includegraphics[width=\textwidth]{images/gibbs-4}
        \label{fig:fib-1}
    \end{subfigure}
    ~
    \begin{subfigure}[H]{0.30\textwidth}
        \includegraphics[width=\textwidth]{images/gibbs-5}
        \label{fig:fib-2}
    \end{subfigure}
    ~
    \begin{subfigure}[H]{0.30\textwidth}
        \includegraphics[width=\textwidth]{images/gibbs-6}
        \label{fig:fib-2}
    \end{subfigure}        
    
    \caption{Voltage curves are linearly related to the slope of the free energy of the electrode material}

\end{figure}

\begin{equation*}
V_{cell} = \frac{\mu_{Li}^{cathode} - \mu_{Li}^{anode}}{e} \qquad \mu_{Li} = \frac{\delta g}{\delta x}
\end{equation*}

\end{frame}

\section{Cell Setup}

%%%%%%%%%%%%%%%%%%%%%%% Swagelok and Beaker Cell %%%%%%%%%%%%%%%%%%%%%%%%%%%

\begin{frame}{Swagelok and Beaker Cell}

\begin{figure}[H]
    \centering
    \begin{subfigure}[H]{0.44\textwidth}
        \includegraphics[width=\textwidth]{images/swagelok}
        \caption{Swagelok}
    \end{subfigure}
    ~
    \begin{subfigure}[H]{0.44\textwidth}
        \includegraphics[width=\textwidth]{images/beaker-cell.png}
        \caption{Beaker Cell}
    \end{subfigure}
    \caption{Cell set up for bulk and thin film Al half cells}
   
\end{figure}

\end{frame}


%%%%%%%%%%%%%%%%%%%%%%% Initial lithiation of bulk Al (poly) %%%%%%%%%%%%%%%%%%%%%%%%%%%

\begin{frame}[fragile]{Initial lithiation of bulk Al (poly)}
Bulk Al electrode(12 mm dia * 3 mm thickness) was galvanostatically lithiated at 25 $\mu$A/cm$ ^{2} $, in 1 M lithium hexafluoro phosphate, 1:1 mixture of ethylene carbonate and dimethylene carbonate. 

\begin{figure}[H]
\includegraphics[width=\textwidth,height=0.6\textheight,keepaspectratio]{images/arbin-1}
\end{figure}

\end{frame}

%%%%%%%%%%%%%%%%%%%%%%% Initial lithiation/delithiation of bulk Al (poly) %%%%%%%%%%%%%%%%%%%%%%%%%%%

\begin{frame}[fragile]{Initial lithiation/delithiation of bulk Al (poly)}

\begin{figure}[H]
\includegraphics[width=\textwidth,height=0.6\textheight,keepaspectratio]{images/arbin-2}
\end{figure}
Unlike silicon, Al phase transition happens reversibly!
First cycle reversibility is good (can be improved with proper potential  limits during cycling)
The appearance of nucleation peak at the beginning of phase formation

\end{frame}

%%%%%%%%%%%%%%%%%%%%%%% Cycling of bulk Al (poly) half cell %%%%%%%%%%%%%%%%%%%%%%%%%%%

\begin{frame}[fragile]{Cycling of bulk Al (poly) half cell}
\begin{figure}[H]
\includegraphics[width=\textwidth,height=0.75\textheight,keepaspectratio]{images/arbin-5}
\end{figure}

\end{frame}

%%%%%%%%%%%%%%%%%%%%%%% Nucleation Peak Formation %%%%%%%%%%%%%%%%%%%%%%%%%%%

\begin{frame}[fragile]{Nucleation Peak Formation}
Galvanostatic cycling at 25 $\mu$A/cm$^{2}$.
\begin{figure}[H]
    \centering
    \begin{subfigure}[H]{0.40\textwidth}
        \includegraphics[width=\textwidth]{images/arbin-3}
        \caption{Cut-off potential of 1.2 V vs. Li/Li+ ie; fully discharged}
       
    \end{subfigure}
    ~
    \begin{subfigure}[H]{0.40\textwidth}
        \includegraphics[width=\textwidth]{images/arbin-4}
        \caption{Cut-off potential of 0.5 V vs. Li/Li$ ^{+} $ ie; partially discharged}
      
    \end{subfigure}
  
\end{figure}
Anomalies observed:  
\begin{enumerate}
\item The appearance of nucleation peak, even after few initial cycles. 
\item The appearance of nucleation peak, even if lithium was not fully extracted from the electrode during cell discharge.
\end{enumerate}


\end{frame}        

%%%%%%%%%%%%%%%%%%%%%%% Phase Analysis by XRD %%%%%%%%%%%%%%%%%%%%%%%%%%%

{\setbeamertemplate{frame footer}{LiAl reference: ICSD CC No.1924 
Kuriyama K, Masaki N, ``The crystal structure of LiAl'', Acta Cryst. B, 1975, 31 1793}
\begin{frame}[fragile]{Phase Analysis by XRD}

\begin{figure}[H]
    \centering
    \begin{subfigure}[H]{0.45\textwidth}
        \includegraphics[width=\textwidth]{images/arbin-1}
        %\caption{Initial Lithiation of bulk Al}
    \end{subfigure}
    ~
    \begin{subfigure}[H]{0.45\textwidth}
        \includegraphics[width=\textwidth]{images/Sputtering}
        \caption{Protective Alumina}
    \end{subfigure}
    \\
    \begin{subfigure}[H]{0.45\textwidth}
        \includegraphics[width=\textwidth]{images/LiAl-xrd}
        %\caption{Phases formed after lithiation}
    \end{subfigure}    
  
\end{figure}
\end{frame}
}

%%%%%%%%%%%%%%%%%%%%%%% Microstructural Characterization Of LiAl phase %%%%%%%%%%%%%%%%%%%%%%%%%%%

\begin{frame}[fragile]{Microstructural Characterization Of LiAl phase}

\begin{figure}[H]
    \centering
    \begin{subfigure}[H]{0.40\textwidth}
        \includegraphics[width=\textwidth]{images/sem-1}
  
    \end{subfigure}
    ~
    \begin{subfigure}[H]{0.40\textwidth}
        \includegraphics[width=\textwidth]{images/sem-2}
        
    \end{subfigure}
    \\
    \begin{subfigure}[H]{0.40\textwidth}
        \includegraphics[width=\textwidth]{images/sem-3}
      
    \end{subfigure}  
    ~
    \begin{subfigure}[H]{0.40\textwidth}
        \includegraphics[width=\textwidth]{images/sem-4}
    \end{subfigure}       
    
    \caption{SEM image of FIB milled Thin Film Aluminium after electrochemical lithiation}
  
\end{figure}

\end{frame}

%%%%%%%%%%%%%%%%%%%%%%% Microstructural Characterization Of LiAl phase %%%%%%%%%%%%%%%%%%%%%%%%%%%

\begin{frame}[fragile]{Microstructural Characterization Of LiAl phase}

\begin{figure}[H]
    \centering
    \begin{subfigure}[H]{0.40\textwidth}
        \includegraphics[width=\textwidth]{images/sem-cs-1}
  
    \end{subfigure}
    ~
    \begin{subfigure}[H]{0.40\textwidth}
        \includegraphics[width=\textwidth]{images/sem-cs-2}
        
    \end{subfigure}
    \\
    \begin{subfigure}[H]{0.40\textwidth}
        \includegraphics[width=\textwidth]{images/sem-cs-3}
      
    \end{subfigure}  
    ~
    \begin{subfigure}[H]{0.40\textwidth}
        \includegraphics[width=\textwidth]{images/sem-cs-4}
    \end{subfigure}       
    
    \caption{Cross-sectional view of the negative electrode}
  
\end{figure}

\end{frame}

%%%%%%%%%%%%%%%%%%%%%%% Qualitative Texture analysis of lithiated bulk aluminium %%%%%%%%%%%%%%%%%%%%%%%%%%%

\begin{frame}[fragile]{Qualitative Texture analysis of lithiated bulk aluminium}

\begin{figure}[H]
    \centering
    \begin{subfigure}[H]{0.35\textwidth}
        \includegraphics[width=\textwidth]{images/ebsd-1.png}
        \caption{Area Selected for EBSD}
    \end{subfigure}
    ~
    \begin{subfigure}[H]{0.35\textwidth}
        \includegraphics[width=\textwidth]{images/ebsd-2.png}
        \caption{Lithiated aluminium}
    \end{subfigure}
    \\
    \begin{subfigure}[H]{0.65\textwidth}
        \includegraphics[width=\textwidth]{images/ebsd-3.png}
        \caption{EBSD image of selected area}
    \end{subfigure}    
    
    \caption{Qualitative Texture Analysis}
\end{figure}

\end{frame}

%%%%%%%%%%%%%%%%%%%%%%% Initial lithiation of aluminum thin film half cell %%%%%%%%%%%%%%%%%%%%%%%%%%%

\begin{frame}[fragile]{Initial lithiation of aluminum thin film half cell}

100 nm Al thin-film electrode galvanostatic lithiation/delithiation at 2.5 $\mu$A/cm$ ^{2} $ (between 0.01 V and 1 V), in 1 M lithium hexafluoro phosphate, 1:1 mixture of ethylene carbonate and dimethylene carbonate. 

\begin{figure}[H]
\includegraphics[width=\textwidth,height=0.75\textheight,keepaspectratio]{images/arbin-7}
\end{figure}
The film continues to lithiate  beyond the LiAl phase (a solid solution of LiAl + a new phase)?
\end{frame}

%%%%%%%%%%%%%%%%%%%%%%% Electrochemical cycling of aluminium thin films %%%%%%%%%%%%%%%%%%%%%%%%%%%

\iffalse

\begin{frame}[fragile]{Electrochemical cycling of aluminium thin films}
\begin{figure}[H]
\includegraphics[width=\textwidth,height=0.75\textheight,keepaspectratio]{images/arbin-6}
\end{figure}

\end{frame}

\fi

\section{Stresses in cells during operation}

%%%%%%%%%%%%%%%%%%%%%%% Experimental system for studying electrode mechanics %%%%%%%%%%%%%%%%%%%%%%%%%%%

{\setbeamertemplate{frame footer}{G.G. Stoney, Proc. R. Soc. A, 1909, 82, 172}
\begin{frame}[fragile]{Experimental system for studying electrode mechanics}

\begin{figure}[H]
\includegraphics[width=\textwidth,height=0.65\textheight,keepaspectratio]{images/Thin-Film-Bending}
\end{figure}
~
\begin{equation*}
\sigma = \frac{M_{s}h_{s}^{2}}{6h_{f}}\frac{1}{R} 
\label{eqn:stoney}
\end{equation*}

\end{frame}
}
%%%%%%%%%%%%%%%%%%%%%%% Experimental setup for curvature measurements %%%%%%%%%%%%%%%%%%%%%%%%%%%

\begin{frame}{Experimental setup for curvature measurements}

\begin{figure}[H]
\centering
\includegraphics[width=\textwidth,height=0.75\textheight,keepaspectratio]{images/MOSS-combined}
\caption{Multi-beam Optical Stress Sensor Assembly}
\end{figure}

\end{frame}

%%%%%%%%%%%%%%%%%%%%%%% Rough estimation of Volume expansion of LiAl Phase formed %%%%%%%%%%%%%%%%%%%%%%%%%%%

\begin{frame}[fragile]{Rough estimation of Volume expansion of LiAl Phase formed}
\begin{figure}[H]
\centering
\includegraphics[width=\textwidth,height=0.75\textheight,keepaspectratio]{images/sem-hf}
\end{figure}
\begin{equation*}
        V_{o} = V_{c} + V_{e}
\end{equation*}
\end{frame}

%%%%%%%%%%%%%%%%%%%%%%% Volume expansion during LiAl phase formation %%%%%%%%%%%%%%%%%%%%%%%%%%%

\begin{frame}{Volume expansion during LiAl phase formation}

\begin{figure}[H]
    \centering
    \begin{subfigure}[H]{0.44\textwidth}
        \includegraphics[width=\textwidth]{images/FIB-1.png}
        \caption{Top View of FIB Milled Al}
        \label{fig:fib-1}
    \end{subfigure}
    ~
    \begin{subfigure}[H]{0.44\textwidth}
        \includegraphics[width=\textwidth]{images/FIB-2.png}
        \caption{$\beta$-LiAl Film Thickness}
        \label{fig:fib-2}
    \end{subfigure}
    \caption{SEM image of FIB milled Thin Film Aluminium after electrochemical lithiation}
  
\end{figure}

Volume expansion calculated to be 89.6\%, assuming capacity to be 993 mAh/g for LiAl excluding SEI formation.
\end{frame}

%%%%%%%%%%%%%%%%%%%%% Stress evolution during lithiation/delithiation of Al thin film %%%%%%%%%%%%%%%%%%%%%%%%

\begin{frame}{Stress evolution during lithiation/delithiation of Al thin film}

\begin{wrapfigure}{l}{0.5\textwidth}
\vspace{-25pt}
\begin{figure}[!htbp]
\centering
\includegraphics[width=0.48\textwidth,keepaspectratio]{images/stress-graph}
\caption{Stresses evolved during electrochemical cycling}
\end{figure}
\end{wrapfigure}

Converting stress-thickness data  to biaxial film-stress involves accounting for :
\begin{enumerate}
\item volume expansion during initial lithium solubility in Al;(Neglected in this study).

\item volume expansion during LiAl phase formation; and calculated to be 87%.

\item Height is assumed to vary linearly as: h=h$ _{f0} $ (1+0.89z)

\item volume expansion during formation of solid-solution of LiAl and Li$ _{3} $Al$ _{2} $ phases (Neglected in this study)
\end{enumerate}

\end{frame}

%%%%%%%%%%%%%%%%%%%%%%% Fracture Energy of LiAl %%%%%%%%%%%%%%%%%%%%%%%%%%%
\iffalse
{\setbeamertemplate{frame footer}{J. L. Beuth, JR. 'Cracking of thin bonded films in residual tension' J. Solid Structures, 1990, 29, 1657-1675}
\begin{frame}{Fracture Energy of LiAl}

\begin{wrapfigure}{l}{0.5\textwidth}
\begin{figure}[!htbp]
\centering
\includegraphics[width=0.48\textwidth,height=\textheight,keepaspectratio]{images/arbin-9}
\caption{Stresses evolved during electrochemical cycling}
\end{figure}
\end{wrapfigure}

Peak tensile stress 0.9 Gpa, the films were found to be still intact, devoid of cracks.
\\
Equating G = $\tau$
\begin{equation*}
G = \frac{\pi}{2} \frac{(1-\nu_{f}^{2})t\_{f}\sigma^{2}}{E\_{f}}g(\alpha,\beta) 
\end{equation*}

Height is assumed to vary  linearly:
h$_{f}$ = h$_{f0}$ (1+0.896z)
      
Elastic Modulus of LiAl  (Nano Indentation Data) ca. 46 GPa. Fracture energy for channeling crack of lithiated Al is calculated to be 4.2 J/m$ ^{2} $. Peculiar behaviour of plastic flow with brittle fracture.
\end{frame}
}
\fi

%%%%%%%%%%%%%%%%%%%%%%% Conclusion %%%%%%%%%%%%%%%%%%%%%%%%%%%
 
\section{Conclusion}
\begin{frame}{Insights gained thus far...}

\begin{enumerate}

\item Electrochemical curves conformed to thermodynamic studies.

\item Stability of the electrode was studied.

\item Formation of $\beta$-LiAl phase on the negative electrode.

\item Al - Li system showed reversible phase transformation.

\item The percentage straining due to the phase transformation was 89.6\%.

\item Microstructural characterization of Al-Li electrode revealed non-uniform LiAl phase growth.

\item SEM EBSD texture studies revealed orientation dependence of LiAl phase growth.

\item In situ stress measurements using MOSS and Stoney's equation showed that, Al underwent extensive plastic deformation with a flow stress in the range of 1.2 - 1.5 GPa in compression (during lithiation) and 0.7 - 1 GPa in tension (during delithiation).

\end{enumerate}

\end{frame}

%%%%%%%%%%%%%%%%%%%%%%% Scope for future work %%%%%%%%%%%%%%%%%%%%%%%%%%%

\begin{frame}{Scope for future work}
\begin{enumerate}
\item Al-Li half cell showed good phase reversibility during eletrochemical cycling. So Lithium ion full cells with Al as negative electrode can be studied. 
\item Studies on other phases formed on negative electrode during electrochemical cycling and their corresponding effect on half cell performance can be done.
\item Studies on single crystal aluminium as negative electrode can be done.
\end{enumerate}
\end{frame}

%%%%%%%%%%%%%%%%%%%%%%% Thank You %%%%%%%%%%%%%%%%%%%%%%%%%%%

\begin{frame}[standout]
  Thank You
\end{frame}

\end{document}
